\section{Σημασιολογική ανάλυση}
Μια άλλη κατεύθυνση που συναντάται στη βιβλιογραφία για σκοπούς σχετικούς με αυτούς της παρούσας διπλωματικής σχετίζεται με τη \newterm{Σημασιολογική ανάλυση}{Semantic Parsing} κειμένων.

\subsection{Ρομποτικές εφαρμογές}
Το πρόβλημα της \newterm{Θεμελίωση\dd{ς} Φυσικής Γλώσσας}{Grounding Natural Language} σε έρευνες του τομέα της ρομποτικής συνήθως επιλύεται λαμβάνοντας υπόψη το περιβάλλον του ρομπότ.
Αυτό αποτελεί σημαντική διαφοροποίηση από τους στόχους αυτής της εργασίας όπου γίνεται προσπάθεια γενίκευσης του προβλήματος της παραγωγής κώδικα ρομποτικής εφαρμογής,
διατηρώντας ως κοινή αναφορά μόνο το μέτα-μοντέλο \metamodel{}.

Πολλές δημοσιεύσεις επιχειρούν τη μετατροπή εντολών φυσικού κειμένου στη \newtermprint[Robot Control Language - RCL]{Γλώσσα Ρομποτικού Ελέγχου}.
Στο~\cite{matuszek2013learning} γίνεται εκμάθηση ενός αναλυτή μέσω της εκπαίδευσης σε ζευγάρια εντολών γραμμένες στα αγγλικά και της μετάφρασής τους σε RCL.
Τα κείμενα αφορούν οδηγίες δρομολόγησης ενός ρομπότ μέσα σε έναν προηγουμένως άγνωστο κλειστό χώρο.
Χρησιμοποιείται \newtermprint[Combinatory Categorial Grammar - CCG]{Συνδυαστική κατηγοριακή γραμματική} που μοντελοποιεί ταυτόχρονα τη συντακτική και σημασιολογική δομή μιας πρότασης.
Οι \citet{shimizu2009learning} εκπαιδεύουν ένα Μαρκοβιανό μοντέλο κατανόησης φυσικής γλώσσας για την ανάλυση οδηγιών δρομολόγησης.
Επίσης, έχουν χρησιμοποιηθεί μέθοδοι ενισχυτικής μάθησης~\cite{branavan2009reinforcement}
και πιθανοτικοί γράφοι~\cite{tellex2011understanding,tellex2011approaching}
για την αντιστοίχιση γλώσσας σε εντολές.

\subsection{Άλλες εφαρμογές}
Η σημασιολογική ανάλυση μπορεί να χρησιμοποιηθεί στην παραγωγή κώδικα δεδομένης μιας λεκτικής περιγραφής.
Στο~\cite{quirk2015language} γίνεται εξαγωγή \enquote{συνταγών} που χρησιμοποιούνται στη διαδικτυακή υπηρεσία \lib{IFTTT}\footnote{\url{https://ifttt.com/}}.
Τα~\cite{yin2017syntactic,rabinovich2017abstract} εκμεταλλεύονται νευρωνικά δίκτυα για την παραγωγή κώδικα \lib{Python} από φυσική γλώσσα.

\subsubsection{Παράδειγμα από τεχνολογία λογισμικού}
Οι \citet{diamantopoulos2017software} προσπαθούν να αυτοματοποιήσουν την αντιστοίχιση λειτουργικών απαιτήσεων,
που γράφονται σε φυσική γλώσσα στα πρώιμα στάδια της ανάπτυξης λογισμικού,
σε \newtermprint[Formal Specification]{τυπικές προδιαγραφές}.
Δεδομένης μιας πρότασης που περιέχει μια λειτουργική απαίτηση, γίνεται εξαγωγή των οντοτήτων και των σχέσεων μεταξύ τους,
είτε αυτών που εκφράζονται ρητά είτε εκείνων που μπορούν να συναχθούν.

Το μοντέλο τους περιλαμβάνει μια σωλήνωση λογισμικού που εκτελεί συντακτική ανάλυση, σημασιολογική ανάλυση, εξαγωγή χαρακτηριστικών και εκπαίδευση ταξινομητών λογιστικής παλινδρόμησης.
Κατά τη σημασιολογική ανάλυση, γίνεται κατηγοριοποίηση των όρων του κειμένου στις αντίστοιχες οντολογικές έννοιες ή ιδιότητες.

Αν και βρίσκεται αρκετά κοντά στους στόχους αυτής της διπλωματικής εργασίας, θεωρείται ότι οι προτάσεις των λειτουργικών απαιτήσεων είναι διατυπωμένες σε σχετικά επίσημη και δομημένη γλώσσα.
Επίσης, η οντολογία τους είναι περιορισμένη και περιγράφει μόνο γενικές έννοιες όπως για παράδειγμα \enquote{\entity{Πράκτορας}} (\entity{Actor}), \enquote{\entity{Αντικείμενο}} (\entity{Object}) κ.ά.,
με περιορισμένο αριθμό υποκλάσεων.

% vim:ts=4:sw=4:expandtab:fo-=tc:tw=120
