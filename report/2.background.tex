\chapter{Υπόβαθρο}\label{chap:background}

Στο κεφάλαιο αυτό παρουσιάζεται το θεωρητικό υπόβαθρο που σχετίζεται με την παρούσα εργασία.
Περιγράφονται περιληπτικά βασικές ιδέες και έννοιες σχετικές με τη μηχανική εκμάθηση και την κατανόηση φυσικής γλώσσας.

Η έννοια της μάθησης συνδέεται με την ικανότητα ενός προγράμματος να βελτιώνει με την επανάληψη και την πρόσκτηση επιπλέον γνώσης την απόδοσή του.
Αποτελεί πεδίο μελέτης που δίνει στους υπολογιστές την ικανότητα να μαθαίνουν, χωρίς να έχουν ρητά προγραμματιστεί~\cite{samuel1959some}.
Στο~\cite{mitchell1990machine}, η \newterm{Μηχανική Εκμάθηση}{Machine Learning - ML} ορίζεται πιο συγκεκριμένα ως:
\begin{framed}
    Λέμε ότι ένα πρόγραμμα μαθαίνει από την υπάρχουσα εμπειρία $E$ αναφορικά με κάποια εργασία $T$ που πρέπει να επιτελέσει και με κάποιο μέτρο απόδοσης $P$,
    όταν η απόδοση του στην εργασία $T$, όπως μετριέται από το $P$, βελτιώνεται από την εμπειρία $E$.%
\end{framed}

Παραδοσιακά, η μηχανική εκμάθηση μπορεί να χωριστεί σε τρεις κατηγορίες~\cite{Goodfellow-et-al-2016}:
\begin{enumerate}
    \item \newtermc{Μάθηση με Επίβλεψη}{Supervised Learning}:
          Οι αλγόριθμοι μάθησης με επίβλεψη αποκτούν εμπειρία πάνω σε ένα σύνολο δεδομένων που περιέχει διάφορα χαρακτηριστικά και κάθε στοιχείο του συνοδεύεται από κάποια ετικέτα ή στόχο.
          Ονομάζονται έτσι λόγω του ότι η έξοδος $\vy$ του αλγόριθμου παρέχεται από κάποιο \enquote{δάσκαλο} που δείχνει στο σύστημα μηχανικής εκμάθησης τι να κάνει.
          Οι αλγόριθμοι αυτοί παρατηρούν πολλές πιθανές τιμές ενός τυχαίου διανύσματος $\vx$ και το αντίστοιχο διάνυσμα εξόδου $\vy$ και μαθαίνουν πώς να προβλέπουν το $\vy$ από το $\vx$,
          συνήθως υπολογίζοντας την υπό συνθήκη πιθανότητα $p(\vy \mid \vx)$.

          Ανάλογα με τον τύπο της εξόδου $\vy$, συνήθως τα προβλήματα της μάθησης με επίβλεψη διακρίνονται σε αυτά της παλινδρόμησης και αυτά της ταξινόμησης.
          \begin{itemize}
              \item Στην \newterm[Μάθηση με Επίβλεψη - Supervised Learning]{Παλινδρόμηση}{Regression}
                    ζητείται από τον αλγόριθμο εκμάθησης να εξάγει μία συνάρτηση $f : \mathbb{R}^n \mapsto \mathbb{R}$
                    (ή κάποιο άλλο συνεχές υποσύνολο του $\mathbb{R}$),
                    δηλαδή να υπολογίζεται μια αριθμητική τιμή δεδομένου του διανύσματος εισόδου.
                    Παράδειγμα τέτοιου προβλήματος αποτελεί η πρόβλεψη της ζήτησης του ηλεκτρικού φορτίου σε ένα δίκτυο διανομής ηλεκτρικής ενέργειας.

                    Όταν η συνάρτηση $f : \mathbb{R}^n \mapsto \mathbb{R}^k$ αντιστοιχίζει κάθε τιμή εισόδου σε ένα διάνυσμα $\vy$ διάστασης $k$,
                    το πρόβλημα ονομάζεται \newterm{Παλινδρόμηση Πολλών Εξόδων}{Multioutput Regression}.

              \item Στην \newterm[Μάθηση με Επίβλεψη - Supervised Learning]{Ταξινόμηση}{Classification}
                    ή \newtermsee[Μάθηση με Επίβλεψη - Supervised Learning]{Κατηγοριοποίηση}{}{Ταξινόμηση}
                    το μοντέλο αντιστοιχίζει την είσοδο σε μια διακριτή τιμή εξόδου.
                    Συνήθως, παράγεται μια συνάρτηση $f : \mathbb{R}^n \mapsto \{1 \ldots k\}$ της οποίας η είσοδος αντιστοιχίζεται μια κατηγορία ή κλάση.
                    Όταν υπάρχουν μόνο δύο πιθανές κλάσεις, $k = 2$, το πρόβλημα ονομάζεται
                    \newterm[Μάθηση με Επίβλεψη - Supervised Learning!Ταξινόμηση - Classification]{Δυαδική\dd{ς ταξινόμησης}}{Binary\dd{ Classification}}.
                    Αντίθετα, όταν υπάρχουν περισσότερες από δύο κλάσεις, $k > 2$, το πρόβλημα ονομάζεται
                    \newterm[Μάθηση με Επίβλεψη - Supervised Learning!Ταξινόμηση - Classification]{Πολλών Κλάσεων}{Multiclass\dd{ Classification}}
                    ή \newtermsee[Μάθηση με Επίβλεψη - Supervised Learning!Ταξινόμηση - Classification]{Πολυωνυμική\dd{ς} Ταξινόμηση\dd{ς}}{Multinomial\dd{ Classification}}{Πολλών Κλάσεων}.
                    Πολλοί ταξινομητές είναι εκ φύσεως δυαδικοί και απαιτούν διάφορες στρατηγικές για τη μετατροπή τους σε ταξινομητές πολλαπλών κλάσεων.

                    Επίσης, όταν η συνάρτηση $f : \mathbb{R}^n \mapsto 2^{\{1 \ldots k\}}$ αντιστοιχίζει κάθε τιμή εισόδου σε ένα υποσύνολο του συνόλου κλάσεων $\{1 \ldots k\}$, το πρόβλημα ονομάζεται
                    \newterm[Μάθηση με Επίβλεψη - Supervised Learning!Ταξινόμηση - Classification]{Πολλών Ετικετών}{Multi-Label\dd{ Classification}}.
                    Μπορούμε να πούμε ότι η $f : \mathbb{R}^n \mapsto {\{0, 1\}}^k$ αντιστοιχίζει την είσοδο σε ένα $k$-διάστατο δυαδικό διάνυσμα $y$ όπου κάθε στοιχείο του αντιστοιχεί σε μια από τις $k$ ετικέτες.
          \end{itemize}

    \item \newtermc{Μάθηση χωρίς Επίβλεψη}{Unsupervised Learning}:
          % Unsupervised learning algorithms experience a dataset containing many features, then learn useful properties of the structure of this dataset.
          Οι αλγόριθμοι μάθησης χωρίς επίβλεψη αποκτούν εμπειρία πάνω σε ένα σύνολο δεδομένων που περιέχει διάφορα χαρακτηριστικά και μαθαίνουν χρήσιμες ιδιότητες σχετικά με τη δομή αυτού του συνόλου δεδομένων.
          % In unsupervised learning, there is no instructor or teacher, and the algorithm must learn to make sense of the data without this guide.
          Σε αυτόν τον τύπο μάθησης, δεν υπάρχει \enquote{δάσκαλος} και ο αλγόριθμος πρέπει να μάθει να βγάζει νόημα από τα δεδομένα χωρίς οδηγό.
          % Roughly speaking, unsupervised learning involves observing several examples of a random vector x, and attempting to implicitly or explicitly learn the probability distribution p(x), or some interesting properties of that distribution.
          Παρατηρεί πολλές πιθανές τιμές ενός τυχαίου διανύσματος $x$ και προσπαθεί να μάθει την κατανομή πιθανότητας $p(x)$ ή κάποιες άλλες χρήσιμες ιδιότητές της.

    \item \newtermc{Ενισχυτική Μάθηση}{Reinforcement Learning}:
          Ένα πρόγραμμα υπολογιστή αλληλεπιδρά με ένα δυναμικό περιβάλλον στο οποίο πρέπει να επιτευχθεί ένας συγκεκριμένος στόχος (όπως η οδήγηση ενός οχήματος ή η νίκη σε μια παρτίδα σκάκι),
          χωρίς κάποιος δάσκαλος να του λέει ρητά αν έχει φτάσει κοντά στον στόχο του αλλά μέσω της μεθόδου της \enquote{δοκιμής και λάθους}.
\end{enumerate}

Από εδώ και στο εξής θα αναφερόμαστε μόνο σε αλγορίθμους μηχανικής μάθησης με επίβλεψη εκτός και αν αναφέρεται ρητά το αντίθετο.

\section{Εκπαίδευση και αξιολόγηση}
Ένας αλγόριθμος μηχανικής εκμάθησης αποκτά εμπειρία σε ένα \newtermprint[dataset]{Σύνολο Δεδομένων}.
Τα δεδομένα που χρησιμοποιούνται για την εκπαίδευση του αλγορίθμου ονομάζονται \newterm{Σύνολο Εκπαίδευσης}{Training Set} και αξιοποιούνται για να προσαρμοστούν οι παράμετροι του μοντέλου.

Μια από τις σημαντικότερες προκλήσεις στον τομέα της μηχανικής μάθησης είναι η επιθυμία καλών επιδόσεων των μοντέλων ακόμα και σε δεδομένα πάνω στα οποία δεν έχουν εκπαιδευτεί.
Λέμε ότι ένα μοντέλο που μπορεί να αποδίδει καλά σε καινούργια δεδομένα έχει δυνατότητα \newterm{Γενίκευση\dd{ς}}{Generalization}.

Κατά τη διάρκεια της εκπαίδευσης, ο αλγόριθμος μηχανικής εκμάθησης προσπαθεί να ελαχιστοποιήσει κάποια μετρική που μετράει το σφάλμα εξόδου, δηλαδή πραγματοποιεί κάποιο είδος βελτιστοποίησης.
Το σφάλμα που προκύπτει κατά την εκπαίδευση ονομάζεται \newterm{Σφάλμα Εκπαίδευσης}{Training Error}.
Αυτό που διαχωρίζει τη μηχανική εκμάθηση από την απλή βελτιστοποίηση είναι ότι γίνεται και προσπάθεια ελαχιστοποίησης του αναμενόμενου σφάλματος σε νέα δεδομένα που
λέγεται \newterm{Σφάλμα Γενίκευσης}{Generalization Error} ή \newtermsee{Σφάλμα Δοκιμής}{Test Error}{Σφάλμα Γενίκευσης - Generalization Error}~\cite{Goodfellow-et-al-2016}.

Ένα μοντέλο που έχει σημαντικά μεγαλύτερο σφάλμα δοκιμής από σφάλμα εκπαίδευσης λέμε ότι παρουσιάζει \newterm{Υπερπροσαρμογή}{Overfitting},
δηλαδή έχει φτωχή γενίκευση σε νέα δεδομένα.

Θα θεωρήσουμε ότι μια μηχανή μάθησης έχει δύο στάδια λειτουργίας, την εκπαίδευση (training) και την επαγωγή (inference).

\section{Συναρτήσεις κόστους και απωλειών}
Αναφέρθηκε ότι καθώς ο αλγόριθμος προσπαθεί να αποτυπώσει τη συσχέτιση μεταξύ των εισόδων και των εξόδων στο σύνολο εκπαίδευσης, υπολογίζεται η απόδοσή του με τη χρήση κάποιας μετρικής.
Αυτή η μετρική συνήθως λέγεται \newterm{Συνάρτηση Κόστους}{Cost Function}.

Αρχικά, ορίζουμε ως \newterm{Συνάρτηση Απωλειών}{Loss Function} μια συνάρτηση που δέχεται ως είσοδο την πραγματική τιμή της εξόδου $y$ και μια πρόβλεψη $\hat{y}$ και υπολογίζει με κάποιο τρόπο κατά πόσο αυτές οι δύο τιμές διαφέρουν.
Μερικές συχνά χρησιμοποιούμενες συναρτήσεις απωλειών είναι η \newterm[Συνάρτηση Απωλειών - Loss Function]{Τετραγωνική Συνάρτηση Απωλειών}{Quadratic Loss Function}:
\begin{equation}
    \label{eq:quadratic-loss}
    L_{\text{quadratic}} (y, \hat{y}) = C (y - \hat{y})
\end{equation}
η \newterm[Συνάρτηση Απωλειών - Loss Function]{Λογιστική Συνάρτηση Απωλειών}{Logistic Loss Function}:
\begin{equation}
    \label{eq:logistic-loss}
    L_{\text{logistic}} (y, \hat{y}) = \log{(1 + e^{-y \hat{y}})}
\end{equation}
και η \newterm[Συνάρτηση Απωλειών - Loss Function]{Απώλεια Διεντροπίας}{Cross-Entropy Loss}:
\begin{equation}
    \label{eq:cross-entropy-loss}
    L_{\text{cross-entropy}} (y, \hat{y}) = -[y \log{(\hat{y})} + (1 - y)\log{(1 - \hat{y})}]
\end{equation}

Έτσι, μια συνάρτηση κόστους μπορεί να υπολογισθεί ως:
\begin{equation}
    J(\vec{\theta}) = \sum_i (L(h_{\vec{\theta}}(x_i), y_i))
\end{equation}
Όπου:
\begin{conditions}
    m                & ο αριθμός των δειγμάτων στο σύνολο εκπαίδευσης                                                                          \\
    x_i              & το $i$-οστό δείγμα εισόδου του συνόλου εκπαίδευσης                                                                      \\
    y_i              & το $i$-οστό δείγμα εξόδου του συνόλου εκπαίδευσης                                                                       \\
    \vth             & οι επιλεγμένοι παράμετροι του μοντέλου                                                                                  \\
    h_{\vec{\theta}} & η συνάρτηση του μοντέλου δεδομένων των παραμέτρων $\theta$\anoteleia{} αλλιώς λέγεται και \newterm{Υπόθεση}{Hypothesis}
\end{conditions}
Για παράδειγμα, μια συνάρτηση κόστους που χρησιμοποιείται συχνά για την παλινδρόμηση είναι το \newterm{Μέσο Τετραγωνικό Σφάλμα}{Mean Squared Error - MSE}:
\begin{equation}
    \label{eq:MSE}
    J_{\text{MSE}}(\vec{\theta}) = \frac{1}{m} \sum_i(h_{\vec{\theta}}(x_i) - y_i)^2
\end{equation}

Δεδομένης μιας συνάρτησης κόστους $J(\vth)$, μπορούμε να χρησιμοποιήσουμε μια μέθοδο βελτιστοποίησης για την επαναληπτική αναβάθμιση των παραμέτρων $\vth$ του μοντέλου.
Μια τέτοια μέθοδος είναι αυτή της
\newterm{Κατάβαση\dd{ς} Κλίσης}{Gradient Descent}
(ή \newtermsee{Μέγιστης Καθόδου}{Steepest Descent}{Κατάβαση Κλίσης - Gradient Descent})~\cite{boyd2004convex,rovithakis}:
\begin{equation}
    \vth \leftarrow \vth - \alpha \nabla_{\vth}{J(\vth)}
\end{equation}
όπου $a$ ο \newterm{Ρυθμός Μάθησης}{Learning Rate}.
Αναφέρεται και η \newterm{Στοχαστική Κάθοδος Κλίσης}{Stochastic Gradient Descent - SGD} όπου οι παράμετροι ενημερώνονται με κάθε δείγμα στο σύνολο εκπαίδευσης,
αντί να υπολογίζεται η κλίση $\nabla_{\vth}{J(\vth)}$ σε όλο το σύνολο εκπαίδευσης.
Εναλλακτικά, μπορούμε να υπολογίζουμε την κλίση μόνο σε ένα υποσύνολο
(ονομάζεται \newterm{Μικρο-δέσμη}{Mini-batch})
σε κάθε ενημέρωση των παραμέτρων.

\section{Γραμμική παλινδρόμηση}\label{sec:linear-regression}
\ig{linear-regression}{\lcaption{Παράδειγμα γραμμικής παλινδρόμησης}{%
        Η γραμμική παλινδρόμηση προσαρμόζει το βάρος $w_1$ και την προδιάθεση $b$ έτσι ώστε η γραμμή $y = w_1 x + b$ να περνάει όσο το δυνατό πιο κοντά από τα σημεία του συνόλου εκπαίδευσης.
        Ο κώδικας για την παραγωγή του γραφήματος προσαρμόστηκε από
        \url{https://docs.scipy.org/doc/scipy/reference/generated/scipy.stats.linregress.html}.%
    }%
}
\ig[pos=t!]{log-and-linear-regression}{\lcaption{Παράδειγμα σύγκρισης καμπύλης λογιστικής και ευθείας γραμμικής παλινδρόμησης}{%
        Γραφήματα με δύο παραδείγματα κατανομών Bernoulli.
        Η γραμμική παλινδρόμηση μπορεί να χρησιμοποιηθεί ως ταξινομητής αν συγκρίνουμε την έξοδό του με το κατώφλι $0.5$,
        ταξινομώντας τις τιμές μεγαλύτερες από αυτό το όριο ως $1$ και τις υπόλοιπες ως $0$.

        Η ακρίβεια των μοντέλων αναφέρεται στο σύνολο εκπαίδευσης καθώς στο συγκεκριμένο παράδειγμα δεν μας ενδιαφέρει πώς γενικεύονται τα μοντέλα\anoteleia{}
        τα δεδομένα που χρησιμοποιήθηκαν δεν προέρχονται από κάποιο υπαρκτό πρόβλημα αλλά δημιουργήθηκαν για σκοπούς επίδειξης των μοντέλων.

        (\textit{Αριστερά})
        Τα δεδομένα είναι μια απλή γραμμή με γκαουσιανό θόρυβο.
        Ακρίβεια ταξινομητή λογιστικής παλινδρόμησης στο σύνολο εκπαίδευσης: $94\%$,
        ακρίβεια ταξινομητή γραμμικής παλινδρόμησης στο σύνολο εκπαίδευσης: $89\%$.
        Ο κώδικας για την παραγωγή του γραφήματος προσαρμόστηκε από το
        \url{https://github.com/scikit-learn/scikit-learn/blob/0.20.X/examples/linear_model/plot_logistic.py}.

        (\textit{Δεξιά})
        Σε αυτό το παράδειγμα φαίνεται πως μερικές έκτοπες τιμές μπορούν να επηρεάσουν το μοντέλο γραμμικής παλινδρόμησης.
        Ακρίβεια ταξινομητή λογιστικής παλινδρόμησης στο σύνολο εκπαίδευσης: $100\%$,
        ακρίβεια ταξινομητή γραμμικής παλινδρόμησης στο σύνολο εκπαίδευσης: $73.91\%$.%
    }%
}

Η \newterm{Γραμμική Παλινδρόμηση}{Linear Regression} αποτελεί έναν από τους απλούστερους αλγορίθμους μηχανικής εκμάθησης.
Κατασκευάζει μια συνάρτηση που δέχεται ένα διάνυσμα $\vx \in \mathbb{R}^n$ και προσεγγίζει μια βαθμωτή τιμή $y \in \mathbb{R}$.
Η συνάρτηση αυτή είναι της μορφής:
\begin{equation}
    \label{eq:linear-regression}
    f(\vx) = \vw^\intercal \vx + b
\end{equation}
Το διάνυσμα $\vw \in \mathbb{R}^n$ ονομάζεται \newtermprint[Weight Vector]{διάνυσμα βαρών} και καθορίζει πώς κάθε χαρακτηριστικό $x_i$ επηρεάζει την τελική προσέγγιση.
Ο όρος $b$ ονομάζεται \newterm{Πόλωση}{Bias} ή \newtermsee{Κατώφλι}{Threshold}{Πόλωση - Bias} και με αυτόν οι προβλέψεις του μοντέλου δεν χρειάζεται να περνάνε απαραίτητα από την αρχή των αξόνων $(0, 0)$.
Μπορούμε να γράψουμε τη συνάρτηση απλώς ως $\hth(\vx) = \vth^\intercal \vx$ αν επαυξήσουμε το διάνυσμα $\vx$ με ένα επιπλέον στοιχείο σταθερό με $1$ και προσθέσουμε ακόμα ένα βάρος στην ίδια θέση του $\vw$
(από εδώ και στο εξής $\vth$, το διάνυσμα των παραμέτρων του μοντέλου)
που θα έχει τον ίδιο \enquote{ρόλο} με τον όρο της πόλωσης.

Μπορούμε να βρούμε τις καλύτερες παραμέτρους του μοντέλου αν ελαχιστοποιήσουμε το μέσο τετραγωνικό σφάλμα~\ref{eq:MSE} βρίσκοντας το σημείο που μηδενίζεται η κλίση του:
\begin{equation}
    \nabla_{\vth}(J_{\vth}) = 0
\end{equation}
Η λύση προκύπτει~\cite{Goodfellow-et-al-2016}:
\begin{equation}
    \vth = (\vX^\intercal \vX)^{-1} \vX \vec{y}
\end{equation}

\section{Λογιστική παλινδρόμηση}\label{sec:logistic-regression}
\ig{sigmoid}{\lcaption{Παράδειγμα σιγμοειδούς και Softmax}{%
        (\textit{Αριστερά})
        Η σιγμοειδής συνάρτηση στο διάστημα $[-20, 20]$.
        (\textit{Δεξιά})
        Η Softmax συνάρτηση όπου $\vec{z}$ ένα διάνυσμα 15 στοιχείων, ομοιόμορφα καταναμημένων στο διάστημα $[0, 25]$.%
    }%
}

Η \newtermprint{(διωνυμική)} \newterm{Λογιστική Παλινδρόμηση}{Logistic Regression} ή
\newtermsee{Logit Παλινδρόμηση}{}{Λογιστική Παλινδρόμηση}
προβλέπει την πιθανότητα του διανύσματος εισόδου $\vx$ να αντιστοιχεί σε μία από τις δύο κλάσεις της διχοτομικής εξόδου $y$.
Χρησιμοποιεί την έξοδο της γραμμικής συνάρτησης ως είσοδο στη
\newterm{Σιγμοειδή\rr{}{ς Συνάρτηση}}{Sigmoid\rr{}{ Function}} συνάρτηση (βλέπε και \fref{fig:sigmoid}):
\begin{equation}
    \label{eq:sigmoid}
    \sigma(x) = \frac{1}{1 + e^{-x}}
\end{equation}
Η σιγμοειδής συνάρτηση είναι χρήσιμη γιατί η έξοδός της είναι φραγμένη στο $0$ και $1$.
Παρουσιάζει κορεσμό όταν η είσοδός της είναι πολύ μεγάλη σε απόλυτη τιμή, δηλαδή η έξοδός της παρουσιάζει μικρή ευαισθησία σε μικρές μεταβολές της εισόδου.
Συνδυάζοντας την \fref{eq:linear-regression} με την \fref{eq:sigmoid} προκύπτει η συνάρτηση υπόθεσης:
\begin{equation}
    \label{eq:logistic-regression}
    \hth(\vx) = \frac{1}{1 + e^{-\vth \vx}}
\end{equation}

Στη λογιστική παλινδρόμηση δεν υπάρχει λύση κλειστής μορφής όπως στη γραμμική, αλλά πρέπει να ψάξουμε για τη λύση με μεθόδους βελτιστοποίησης~\cite{Goodfellow-et-al-2016}.

\subsection{Softmax παλινδρόμηση}
Η \newterm{Softmax Παλινδρόμηση}{Softmax Regression}
ή \newtermsee{Λογιστική Παλινδρόμηση Πολλών Κλάσεων}{Multiclass Logistic Regression}{Softmax}
είναι μια μέθοδος ταξινόμησης που γενικεύει τη λογιστική παλινδρόμηση σε προβλήματα πολλών κλάσεων~\cite{tao2015bearing}.
Βασίζεται στη χρήση της συνάρτησης Softmax
που δέχεται ένα διάνυσμα $z \in \mathbb{R}^k$ και το κανονικοποιεί σε μια κατανομή πιθανότητας\footnote{Μετά την εφαρμογή της Softmax, όλα τα στοιχεία βρίσκονται στο διάστημα $(0, 1)$ και το άθροισμά τους είναι $1$.}
(βλέπε και \fref{fig:sigmoid}):
\begin{equation}
    \label{eq:softmax}
    \text{softmax}(\vec{z}_i) = \frac{e^{z_i}}{\sum_j{e^{z_j}}}
\end{equation}
Έτσι, η συνάρτηση υπόθεσης παίρνει τη μορφή~\cite{tao2015bearing}:
\begin{equation}
    \label{eq:softmax-regression}
    \hTh(\vx_i)
    = \begin{bmatrix}
        p(y_i = 1 | \vx_i; \vTh) \\
        p(y_i = 2 | \vx_i; \vTh) \\
        \vdots                   \\
        p(y_i = k | \vx_i; \vTh)
    \end{bmatrix}
    = \frac{1}{\sum_{j=1}^k{{e^{\vth_j^\intercal \vx_i}}}} \begin{bmatrix}
        e^{\vth_1^\intercal \vx_i} \\
        e^{\vth_2^\intercal \vx_i} \\
        \vdots                     \\
        e^{\vth_k^\intercal \vx_i}
    \end{bmatrix}
\end{equation}
Όπου:
\begin{conditions}
    m    & ο αριθμός των δειγμάτων στο σύνολο εκπαίδευσης                       \\
    k    & ο αριθμός των κλάσεων                                                \\
    n    & η διάσταση του διανύσματος εισόδου $\vx$                             \\
    \vTh & ο $k \times (n + 1)$ πίνακας των επιλεγμένων παραμέτρων του μοντέλου
\end{conditions}

\subsection{Στρατηγική ενός εναντίον όλων}
Μια άλλη τεχνική για την εφαρμογή της λογιστικής παλινδρόμησης σε προβλήματα πολλών κλάσεων είναι η
\newterm{Στρατηγική ενός εναντίον όλων}{One-vs-all Strategy}
όπου ένας ταξινομητής λογιστικής παλινδρόμησης (ή και οποιοσδήποτε άλλος δυαδικός ταξινομητής)
χρησιμοποιείται για την πρόβλεψη της πιθανότητας κάθε κλάσης ξεχωριστά, δηλαδή σε αντίθεση με τις υπόλοιπες κλάσεις.

\section{Υπό συνθήκη τυχαίο πεδίο}\label{sec:crf}
\ig[type=tikz]{pgm}{\lcaption{Παράδειγμα πιθανοτικού γράφου}{%
        Κάθε βέλος υποδηλώνει πιθανοτική εξάρτηση.
        Το C εξαρτάται άμεσα από τα A και D, το D από τα A, B και C, το E από το C και το F από τα C και D\@.
        \\Σχέσεις ανεξαρτησίας που προκύπτουν από τον γράφο: \begin{tabular}{>{$}c<{$}}
            (A \perp B)                 \\
            (C \perp B \mid D)          \\
            (E \perp A, B, D, F \mid C) \\
            (F \perp A, B \mid C, D)
        \end{tabular}
        \\Παραγοντοποίηση:
        $ P(A, B, C, D, E, F) = P(A) P(B) P(C, D \mid A, B) P(E \mid C) P(F \mid C, D)$%
    }%
}

Ένα \newterm{Πιθανοτικό Γραφικό Μοντέλο}{Probabilistic Graphical Model - PGM} (βλέπε και \fref{fig:pgm})
μοντελοποιεί μια σύνθετη κατανομή πολλών τυχαίων μεταβλητών ως γινόμενο πολλών τοπικών συναρτήσεων που η καθεμία εξαρτάται από ένα μικρό πλήθος μεταβλητών.
Μας επιτρέπει να περιγράψουμε πώς μια παραγοντοποίηση της πιθανότητας ενός γεγονότος αντιστοιχεί σε ένα σύνολο υπό όρους σχέσεων ανεξαρτησίας που ικανοποιούν την κατανομή.
Διευκολύνει τη μοντελοποίηση πιο σύνθετων προβλημάτων καθώς αυτά συνήθως χαρακτηρίζονται από σημαντικό αριθμό εξαρτημένων και ανεξάρτητων γεγονότων~\cite{sutton2012introduction}.
Τα \newterm{Υπό Συνθήκη Τυχαίο Πεδί\rr{α}{ο}}{Conditional Random Field - CRF}~\cite{lafferty2001conditional} αποτελούν ένα τέτοιο μοντέλο.

Συνήθως χρησιμοποιούνται για την κατασκευή πιθανοτικών μοντέλων που τμηματίζουν ή ταξινομούν σειριακά δεδομένα, όπως κείμενα.
Η χρήση τους στον τομέα της επεξεργασίας φυσικής γλώσσας συνδέεται με την ικανότητά τους να πραγματοποιούν ταξινόμηση εξαρτώμενη από τα συμφραζόμενα.

Ορίζουμε ένα
\newterm[Υπό Συνθήκη Τυχαίο Πεδίο - Conditional Random Field - CRF]{\dd{υπό συνθήκη τυχαίο πεδίο }Γραμμικής Αλυσίδας}{Linear Chain\dd{ CRF}}%
\footnote{Από εδώ και στο εξής θα αναφέρονται ως \enquote{CRF γραμμικής αλυσίδας} ή απλώς \enquote{CRF}}
(\enquote{γραμμικής αλυσίδας} σε αντίθεση με τα \enquote{γενικευμένα} που θα αναφερθούν παρακάτω)
τη διανομή $p(\vy \mid \vx)$ που παίρνει τη μορφή:~\cite{lafferty2001conditional,sutton2012introduction}
\begin{equation}
    \label{eq:linear-crf}
    p(\vy \mid \vx) = \frac{1}{Z(\vx)} \prod_{t=1}^T{\exp\left[\sum_{j=1}^k{\theta_j f_j(y_t, y_{t-1}, \vx_t)}\right]}
\end{equation}
Όπου:
\begin{conditions}
    k        & ο αριθμός των κλάσεων                                            \\
    \vy      & το διάνυσμα εξόδου, μια ακολουθία ετικετών $T$ στοιχείων         \\
    y_t      & η $t$-οστή ετικέτα στο διάνυσμα εξόδου                           \\
    \vx      & η ακολουθία εισόδου, περιλαμβάνει $T$ διανύσματα χαρακτηριστικών \\
    \vx_t    & το $t$-οστό διάνυσμα χαρακτηριστικών                             \\
    \theta_j & το $j$-οστό στοιχείο του διανύσματος παραμέτρων (ή βαρών) $\vth$ \\
    f_j      & η $j$-οστή συνάρτηση χαρακτηριστικών                             \\
    Z(\vx)   & η συνάρτηση κανονικοποίησης που παίρνει τη μορφή:
    \begin{equation}
        Z(\vx) = \sum_{\vy} \prod_{t=1}^T{\exp{\left[\sum_{j=1}^k{\theta_j f_j(y_t, y_{t-1}, \vx_t)}\right]}}
    \end{equation}
\end{conditions}

\ig[pos=t,type=tikz]{linear-crf}{\lcaption{Διαγραμματική αναπαράσταση υπό συνθήκη τυχαίου πεδίου γραμμικής αλυσίδας}{%
        Μια ετικέτα $y_t$ επηρεάζεται μόνο από την προηγούμενη
        αλλά μπορεί να επηρεαστεί από όλα τα $\vx_t$.%
    }%
}

Σε ένα CRF γραμμικής αλυσίδας, βλέπουμε ότι οι συναρτήσεις χαρακτηριστικών εξαρτώνται μόνο από την τρέχουσα και την προηγούμενη ετικέτα.
Αυτός ο περιορισμός δεν ισχύει σε ένα \newterm[Υπό Συνθήκη Τυχαίο Πεδίο - Conditional Random Field - CRF]{Γενικευμένο\dd{ υπό συνθήκη τυχαίο πεδίο}}{General\dd{ CRF}}.
Αποτελεί ένα μη κατευθυνόμενο γραφικό μοντέλο όπου οι κόμβοι του διαιρούνται σε δύο ξένα σύνολα, $\vX$ και $\vY$, και μοντελοποιείται η διανομή $p\left(\vY \mid \vX\right)$.

Η λειτουργία αυτών των μοντέλων για την ταξινόμηση βασίζεται στη χρήση χαρακτηριστικών (\en{Features}) που εξάγονται από τα δεδομένα.
Μια \newterm{Συνάρτηση Χαρακτηριστικών}{Feature Function} $f_j$ εκφράζει κάποιο χαρακτηριστικό ενός στοιχείου της ακολουθίας εισόδου και η έξοδός της είναι δυαδική.
Αν το αντίστοιχο βάρος $\theta_j$ είναι μεγάλο, το χαρακτηριστικό που εκφράζει η συνάρτηση είναι σημαντικό και επηρεάζει πιο έντονα την επιλογή της ετικέτας $y_t$.
Μερικά παραδείγματα συναρτήσεων χαρακτηριστικών σε μια εφαρμογή \newterm{Επισημείωση\dd{ς} Μερών του Λόγου}{Part-of-Speech Tagging}:
\begin{compactitem}
    \item $f_1(y_t, y_{t-1}, \vx_t) = 1$ αν $y_{t-1}$ ουσιαστικό και $y_t$ ρήμα
    \item $f_2(y_t, y_{t-1}, \vx_t) = 1$ αν $y_t$ επίρρημα και η κατάληξη της $t$-οστής λέξης είναι \engquote{-ly}
    \item $f_3(y_t, y_{t-1}, \vx_t) = 1$ αν η $t$-οστή λέξη αρχίζει με κεφαλαίο και $y_{t-1}$ σημείο στίξης
\end{compactitem}

Σε αντίθεση με τα μοντέλα \hyperref[sec:linear-regression]{γραμμικής} και \hyperref[sec:logistic-regression]{λογιστικής} παλινδρόμησης,
τα CRF πραγματοποιούν ταξινόμηση εξαρτώμενη από τα συμφραζόμενα, δηλαδή μοντελοποιούν και τις σχέσεις ανάμεσα στα διανύσματα εισόδου.

\section{Νευρωνικά δίκτυα}\label{sec:neural-networks} % TODO
\subsection{Βαθιά μάθηση}\label{subsec:deep-learning} % TODO

% vim:ts=4:sw=4:expandtab:fo-=tc:tw=120

\section{Κατανόηση φυσικής γλώσσας}
\subsection{Όροι γλωσσολογίας}\label{subsec:linguistics}
\begin{itemize}
    \item \newtermprintc{Πρόταση} (\en{Sentence}, \en{Clause} και \en{Proposition}):
          H απόδοση στα ελληνικά των τριών αυτών όρων είναι προβληματική.
          Και οι τρεις μεταφράζονται ως \enquote{πρόταση} αλλά υπάρχουν οι εξής διαφορές:
          \begin{compactitem}
              \item Ο όρος \en{clause} αποδίδεται ως \newtermprint{απλή πρόταση}\index{Απλή Πρόταση - Clause}, π.χ.\ \engquote{The sun is shining}.
              \item Ο όρος \en{sentence} αποδίδεται ως πρόταση αλλά σημαίνει \newtermprint{γραμματική περίοδο}\index{Γραμματική Περίοδος - Sentence},
                    η οποία είναι δυνατό να αποτελείται από περισσότερες επιμέρους προτάσεις ή κώλα,
                    π.χ.\ \engquote{The sun shines when there are no clouds in the sky}.
              \item Ο όρος \en{proposition} αποτελεί έννοια του σημασιολογικού επιπέδου,
                    μιας και δεν αναφέρεται στη γραμματική αλλά στη \newtermprint{λογική πρόταση}\index{Λογική Πρόταση - Preposition}.
                    Για παράδειγμα, οι δομές \engquote{Peter solved the exercise} και \engquote{The exercise was solved by Peter} αποτελούν διαφορετικές γραμματικές προτάσεις αλλά είναι εκδοχές της ίδιας λογικής πρότασης,
                    εφόσον έχουν κοινό λογικό υποκείμενο, τον \engquote{Peter}, και κοινό κατηγόρημα, την επίλυση της άσκησης.
          \end{compactitem}
          Μια πρόταση τυπικά περιλαμβάνει ένα \newterm{Υποκείμενο}{Subject} και ένα \newterm{Κατηγόρημα}{Predicate}.
          Ο όρος κατηγόρημα κατά κύριο λόγο αναφέρεται στο ρήμα της πρότασης και σε εκείνες τις λέξεις που το συνοδεύουν.
    \item \newtermc{Επίρρημα}{Adverb} (από ad (\engquote{to}) + verbum (\engquote{word})):
          λέξη που προσδιορίζει ρήματα, προτάσεις, επίθετα και άλλα επιρρήματα.
          Παραδείγματα:
          \begin{compactitem}
              \item \engquote{\textit{I work often}}: το \engquote{often} προσδιορίζει το ρήμα \engquote{work}.
              \item \engquote{\textit{She was very pretty}}: το \engquote{very} προσδιορίζει το επίθετο \engquote{pretty}.
              \item \engquote{\textit{Often, there have been unacceptable delays}}: το \engquote{often} προσδιορίζει την υπόλοιπη πρόταση.
          \end{compactitem}
    \item \newtermc{Σύνδεσμος}{Conjunction}:
          μέρος του λόγου που χρησιμοποιείται για να συνδέσει λέξεις, φράσεις ή απλές προτάσεις (\en{Clause}).
          Για παράδειγμα, \engquote{I was tired, so I went to bed}: ο σύνδεσμος \engquote{so} ενώνει τις δύο απλές προτάσεις.
    \item \newtermc{Παράταξη}{Coordination}:
          Στην παρατακτική σύνδεση οι προτάσεις παρατάσσονται, δηλαδή μπαίνουν η μία δίπλα στην άλλη και συνδέονται μεταξύ τους με συνδέσμους,
          (συμπλεκτικούς, αντιθετικούς, διαχωριστικούς, συμπερασματικούς κτλ.) που ονομάζονται παρατακτικοί σύνδεσμοι.
    \item \newtermc{Ανύψωση Δεξιού Κόμβου}{Right Node Raising}:
          Ο όρος υποδηλώνει έναν μηχανισμό μοιράσματος με συμμετρικά χαρακτηριστικά,
          σύμφωνα με τον οποίο το υλικό στο δεξί άκρο μιας παράταξης να είναι κατά κάποιον τρόπο κοινό και να ερμηνεύεται ως τμήμα και των δύο συστατικών που συμπλέκονται παρατακτικά.
          Οι παρατάξεις συστηματικά επιτρέπουν μία ακολουθία να προφέρεται μία μόνο φορά, στο δεξί άκρο της παράταξης.
          Για παράδειγμα, \engquote{[Sam likes] but [Fred dislikes] feta cheese}.
    \item \newtermc{Τροποποιητής}{Modifier}\footnote{\url{http://www.greek-language.gr/greekLang/modern_greek/tools/lexica/glossology/show.html?id=333}}~\cite{komvos}:
          Κάθε ονοματική ή ρηματική φράση μπορεί να τροποποιηθεί σημασιολογικά με την χρήση ενός τροποποιητή.
          Οι τροποποιητές είναι προσδιοριστές, δηλαδή συστατικά που περιορίζουν το σημασιακό πεδίο μιας ονοματικής ή ρηματικής φράσης.
          Τα επίθετα, οι μετοχές, οι αναφορικές προτάσεις ή ορισμένες προθετικές φράσεις λειτουργούν ως τροποποιητές ονοματικών φράσεων, ενώ τα επιρρήματα και οι επιρρηματικές δευτερεύουσες προτάσεις είναι τροποποιητές ρηματικών φράσεων.
    \item \newtermc{Αμφισημία}{Ambiguity}~\cite{lexica-glossology}\footnote{\url{http://www.greek-language.gr/greekLang/modern_greek/tools/lexica/glossology/show.html?id=372}}:
          Ο όρος αμφισημία αναφέρεται στην πολλαπλότητα των σημασιών η οποία είναι δυνατό να χαρακτηρίζει μια λέξη ή μια πρόταση.
          Στον πραγματωμένο λόγο, δηλαδή στα φυσικά εκφωνήματα, αίρεται σχεδόν πάντα από τα συμφραζόμενα σε μια δεδομένη περίσταση επικοινωνίας.
          Η συντακτική ή δομική αμφισημία προκύπτει όταν μια πρόταση είναι δυνατόν να αναλύεται από συντακτική άποψη με περισσότερους από έναν τρόπους.
          Για παράδειγμα, η πρόταση \engquote{I saw a girl with a telescope} μπορεί να σημαίνει
          ότι εγώ χρησιμοποίησα τηλεσκόπιο για να δω ένα κορίτσι ή ότι είδα ένα κορίτσι που είχε ένα τηλεσκόπιο.

          Η τέλεια κατανόηση της γλώσσας από έναν υπολογιστή θα απαιτούσε ένα σύστημα τεχνητής νοημοσύνης που θα είχε τη δυνατότητα να επεξεργαστεί όλη την ανθρώπινη γνώση,
          γεγονός που με τη σειρά του θα απαιτούσε ένα σύστημα τεχνητής γενικής νοημοσύνης.
\end{itemize}

\subsection{Επισημείωση μερών του λόγου}
Η \newterm{Επισημείωση Μερών του Λόγου}{Part-of-Speech Tagging} αποτελεί μια βασική διεργασία της επεξεργασίας φυσικής γλώσσας.
Πρόκειται για τη διαδικασία αντιστοίχισης μιας λέξης σε ένα κομμάτι κειμένου με το αντίστοιχο μέρος του λόγου.

Συνήθως χρησιμοποιούνται μέθοδοι επισημείωσης ακολουθιών, όπως τα \CRFR{}.
Η είσοδος $\vx$ που αποτελείται από $T$ λέξεις διαιρείται στα διανύσματα χαρακτηριστικών $\{\vx_0,\allowbreak \vx_1,\allowbreak \ldots,\allowbreak \vx_T\}$
και η έξοδος $y_t$ για την $t$-οστή λέξη αποτελεί το αντίστοιχο μέρος του λόγου.
Κάθε $\vx_t$ μπορεί να περιέχει διάφορες πληροφορίες σχετικά με τη λέξη στη θέση $t$,
όπως την ταυτότητά της, ορθογραφικά χαρακτηριστικά όπως προθέματα και επιθέματα, πληροφορίες σε σημασιολογικές βάσεις δεδομένων όπως το \libcite{WordNet} και το λήμμα της σε ειδικά λεξιλόγια~\cite{sutton2012introduction}.

\subsection{Σημασιολογικοί ρόλοι}\label{subsec:srl}
Η \newterm{Ανάθεση Σημασιολογικών Ρόλων}{Semantic Role Labeling - SRL} στοχεύει να ανακτήσει τη \newterm{Δομή κατηγορημάτων-ορισμάτων}{Predicate-Argument Structure} μιας πρότασης.
Το \newterm{Κατηγόρημα}{Predicate} μιας πρότασης (τυπικά ένα ρήμα) δηλώνει \enquote{τι} έλαβε χώρα και τα άλλα μέρη της πρότασης δηλώνουν τους συμμετέχοντες στο γεγονός (όπως \enquote{ποιος} και \enquote{πού}),
καθώς και άλλα δεδομένα (όπως \enquote{πότε} και \enquote{πώς})~\cite{marquez2008semantic}.
\newtermc{Όρισμα}{Argument} είναι μια έκφραση που βοηθάει στην ολοκλήρωση του νοήματος του κατηγορήματος.

Το πρωταρχικό καθήκον της ανάθεσης σημασιολογικών ρόλων είναι να υποδείξει ποιες σημασιολογικές σχέσεις επικρατούν ανάμεσα σε ένα κατηγόρημα και τα ορίσματά του.
Αυτές οι σχέσεις προέρχονται από έναν προκαθορισμένο κατάλογο πιθανών σημασιολογικών ρόλων για αυτό το κατηγόρημα (ή ομάδα κατηγορημάτων).
Για να επιτευχθεί αυτό, τα συστατικά μιας πρότασης που είναι φορείς ρόλου πρέπει να ταυτοποιηθούν και να τους αποδοθούν οι σωστοί σημασιολογικοί ρόλοι, όπως για παράδειγμα:
\engquote{[The girl on the swing]\textsubscript{Δράστης (Agent)} [whispered]\textsubscript{Κατηγόρημα (Predicate)} to [the boy beside her]\textsubscript{Λήπτης (Recipient)}}.

Ανάθεση σημασιολογικών ρόλων έχει πραγματοποιηθεί σε μεσαία έως μεγάλα σώματα κειμένου στα \libcite{FrameNet} και \libcite{PropBank}, επιτρέποντας έτσι την ανάπτυξη προσεγγίσεων μηχανικής μάθησης.
Το \lib{PropBank} προσθέτει ένα στρώμα σημασιολογικής επισηµείωσης~\cite{kingsbury2002adding} στις συντακτικές δοµές του \lib{Penn Treebank}~\cite{penntreebank}.
% Εκτός από τον σχολιασμό των σημασιολογικών ρόλων, ο σχολιασμός της \lib{PropBank} απαιτεί την επιλογή μιας ταυτότητας (id) νοήματος (γνωστή και ως id frameset ή roleset) για κάθε κατηγόρημα.
Αυτό επιτρέπει επίπεδα σημασιολογικής ανάλυσης που δεν είναι εφικτά μόνο με συντακτική ανάλυση.
Για παράδειγμα, ενώ στις προτάσεις \engquote{John broke the window} και \engquote{The window broke}
η λέξη \engquote{window} είναι το αντικείμενο της πρώτης και στη δεύτερη είναι το υποκείμενο, ο σημασιολογικός του ρόλος είναι ίδιος και στις δύο περιπτώσεις.

Συγκεκριμένα, στο \lib{PropBank}, σε κάθε ρήμα αντιστοιχεί ένα σύνολο σημασιολογικών πλαισίων (αναφέρονται ως \en{Frameset}).
Ο διαχωρισμός γίνεται σύμφωνα με τη σημασία (ρόλο) του ρήματος ανά περίπτωση\anoteleia{} συντακτικές τροποποιήσεις που διατηρούν τη σημασία θεωρούνται ότι ανήκουν στο ίδιο σύνολο.
Κάθε πλαίσιο δέχεται μέχρι και 6 ορίσματα \ARGs{} στα οποία δίνεται μια περιγραφή του ρόλου τους σε σχέση με το κατηγόρημα.
Επίσης οι \newterm{Τροποποιητ\rr{ές}{ής}}{Modifier} ή \newterm{Προσδιορισμ\rr{οί}{ος}}{Adjunct} ονοματίζονται σύμφωνα με τον ρόλο τους,
όπως \engquote{TMP} (\en{Temporal} - Χρονικός) και \engquote{LOC} (\en{Locative} - Χωρικός),
και εφαρμόζονται ισάξια σε κάθε πλαίσιο κάθε ρήματος.
Για παράδειγμα, η επισημείωση της πρότασης \engquote{Mr. Bush met him privately, in the White House, on Thursday} προκύπτει~\cite{propbank2015guidelines}:
\begin{compactitem}
    \item κατηγόρημα: \engquote{met}\footnote{\url{http://verbs.colorado.edu/propbank/framesets-english-aliases/meet.html}}
    \item \enttt{ARG0}: \engquote{Mr. Bush} --- δράστης
    \item \enttt{ARG1}: \engquote{him} --- το πρόσωπο που συναντάει ο δράστης
    \item \enttt{ARGM-MNR}: \engquote{privately} --- ο τρόπος συνάντησης
    \item \enttt{ARGM-LOC}: \engquote{in the White House} --- το μέρος συνάντησης
    \item \enttt{ARGM-TMP}: \engquote{on Thursday} --- ο χρόνος συνάντησης
\end{compactitem}

Η ανάθεση σημασιολογικών ρόλων αποτελεί σημαντικό βήμα για την κατανόηση της σημασίας μιας πρότασης και έχει δειχθεί ότι μπορεί να βελτιώσει αποτελέσματα σε άλλα προβλήματα επεξεργασίας φυσικής γλώσσας
όπως τη \newtermprint[Machine Reading]{Μηχανική Ανάγνωση}~\cite{berant2014modeling,wang2015machine},
την \newtermprint[Question Answering]{Απάντηση Ερωτήσεων}~\cite{narayanan2004question,weston2015towards,yih2016value},
τη \newtermprint[Machine Translation]{Μηχανική Μετάφραση}~\cite{wu2009semantic,liu2010semantic,gao2011utilizing,bazrafshan2013semantic},
τα \newtermprint[Dialogue System]{Σύστημα Διαλόγου}~\cite{tur2005semi,chen2013unsupervised},
και την \newtermprint[Event Extraction]{Εξαγωγή Γεγονότων}~\cite{surdeanu2003using,exner2011using,rospocher2016building}.

\subsection{Συναναφορές}\label{subsec:coref}
Σε ένα κείμενο η \newterm{Συναναφορά}{Coreference} συμβαίνει όταν δύο ή περισσότερες φράσεις αναφέρονται στην ίδια οντότητα (πρόσωπο, τοποθεσία, πράγμα, κτλ).
Για παράδειγμα, στην πρόταση \enquote{Bill said he would come} το κύριο όνομα \engquote{Bill} και η αντωνυμία \engquote{he} αναφέρονται στην ίδια οντότητα, το άτομο που ονομάζεται \en{Bill}.
Αυτό είναι ένα παράδειγμα \newterm[Συναναφορά - Coreference]{Αναφορά\dd{ς}}{Anaphora} καθώς η λέξη \engquote{he} ακολουθεί το όνομα στο οποίο αναφέρεται.
Άλλος τύπος συναναφοράς αποτελεί η \newterm[Συναναφορά - Coreference]{Καταφορά}{Cataphora} όπως στην πρόταση \engquote{If they are angry about the music, the neighbors will call the cops} όπου η καταφορά \engquote{they} προηγείται της φράσης \engquote{the neighbors} στην οποία αναφέρεται.
Άλλες μορφές είναι η \newterm[Συναναφορά - Coreference]{Διάσπαση Προηγουμένων Αναφορών}{Split Antecedents}, π.χ.\ \engquote{When Carol helps Bob and Bob helps Carol, they can accomplish any task} όπου το \engquote{they} αναφέρεται ταυτόχρονα στους \engquote{Bob} και \engquote{Carol}
και η \newterm[Συναναφορά - Coreference]{Συναναφορά Ονοματικών Φράσεων}{Coreferring Noun Phrases}, π.χ.\ \engquote{The project leader is refusing to help. The jerk thinks only of himself} όπου τα \engquote{the project leader} και {the jerk} αναφέρονται στο ίδιο πρόσωπο.

Στην κατανόηση φυσικής γλώσσας, η επίλυση σχέσεων συναναφοράς συνίσταται στην αναζήτηση του ποιες λέξεις αναφέρονται στις ίδιες οντότητες σε ένα κομμάτι κειμένου.
Μια καθολική λύση αποτελεί δύσκολο στόχο καθώς απαιτεί καλή κατανόηση του νοήματος του κειμένου και εξωτερική γνώση.
Η δοκιμασία τεχνητής νοημοσύνης Winograd~\cite{levesque2012winograd} χρησιμοποιεί συναναφορές για να δημιουργήσει προκλήσεις που δεν επιλύονται εύκολα από υπολογιστές.
Για παράδειγμα, στην πρόταση \engquote{The women stopped taking the pills because they were <$X$>} η αναφορά \engquote{they} εξαρτάται από τη λέξη $X$.
Αν $X = \text{\en{pregnant}}$ τότε αναφέρεται στις γυναίκες (\en{the women}) που είναι έγκυες.
Ωστόσο, αν $X = \text{\en{carcinogenic}}$ τότε το \engquote{they} αναφέρεται στα χάπια (\en{pills}) που είναι καρκινογόνα.

% vim:ts=4:sw=4:expandtab:fo-=tc:tw=120


% vim:ts=4:sw=4:expandtab:fo-=tc:tw=120
