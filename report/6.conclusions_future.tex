\chapter{Συμπεράσματα \& Μελλοντική Εργασία}\label{chap:conclusions_future}

\section{Συμπεράσματα}
Σε αυτή τη διπλωματική εργασία παρουσιάστηκε η σωλήνωση λογισμικού \projectname{} που υλοποιεί ένα σύστημα επεξεργασίας και κατανόησης φυσικής γλώσσας.
Στόχος του συστήματος είναι η αναγνώριση και διασύνδεση των ενεργειών και οντοτήτων του μέτα-μοντέλου ενεργειών ρομποτικής πλατφόρμας \metamodel{}.
Το σύστημα εκμεταλλεύεται σχέσεις συναναφοράς που υπάρχουν σε όλο το σώμα του κειμένου εισόδου για να επιλύσει αντωνυμίες όπως το \engquote{it}.
Επίσης, χρησιμοποιεί τους σημασιολογικούς ρόλους για την αναγνώριση πολλών προθέσεων (ενεργειών) ανά πρόταση και τη διασύνδεσή τους μέσω τροποποιητών.

Το σύστημα δεν αναλαμβάνει τη διαδικασία αυτόματης δημιουργίας του κώδικα της ρομποτικής εφαρμογής.
Έτσι, το έργο της παρούσας διπλωματικής δεν είναι άμεσα εκμεταλλεύσιμο από τον τελικό χρήστη,
ωστόσο θεωρείται ότι μπορεί να υλοποιηθεί ένα άλλο σύστημα, ανεξάρτητο από το πλαίσιο αυτής της διπλωματικής, που θα μπορεί να αναλάβει την παραγωγή κώδικα εκμεταλλευόμενο τις πληροφορίες που περιλαμβάνονται στην έξοδο του \projectname{}.
Για παράδειγμα, οι λογικές διακλαδώσεις \engquote{If-else} παραθέτονται σειριακά στην έξοδο του γράφου του τρέχοντος συστήματος, σαν να εξαρτάται η μια από την άλλη.
Το υποτιθέμενο σύστημα για την παραγωγή κώδικα θα πρέπει να τις αναδιατάξει.

Συνοπτικά, το \projectname{} παρουσιάζει:
\begin{itemize}
    \item Μια προσέγγιση διαχωρισμού του προβλήματος σε μονάδες που επιλύουν ξεχωριστά υποπροβλήματα και συνδέονται μέσω μιας διασωλήνωσης λογισμικού.
          Η κάθε μονάδα υλοποιείται με τη χρήση βιβλιοθηκών ελεύθερου λογισμικού, στοχευμένων τόσο για ακαδημαϊκή όσο και για επαγγελματική χρήση, που χρησιμοποιούν σύγχρονες μεθόδους και προσφέρουν εκπαιδευμένα μοντέλα μηχανικής μάθησης.
          Το πλεονέκτημα αυτής της προσέγγισης είναι ότι με τη βελτίωση της απόδοσης αυτών των μοντέλων στο μέλλον το \projectname{} θα παρουσιάσει ανάλογες βελτιώσεις στα αποτελέσματά του.
          Για τις περισσότερες από αυτές τις αλλαγές δεν θα χρειάζονται ραγδαίες τροποποιήσεις στον κώδικα καθώς συνήθως οι βιβλιοθήκες διατηρούν συμβατότητα μεταξύ των διάφορων εκδόσεών τους.

          Επιπλέον, το καθένα από αυτά τα υποπροβλήματα αποτελεί σημαντικό πεδίο έρευνας και η βελτίωση των μοντέλων που το επιλύουν είναι συλλογική διαδικασία.
          Αυτή περιλαμβάνει την ενίσχυση των δεδομένων εκπαίδευσης και τη βελτίωση των αλγορίθμων μηχανικής μάθησης που εφαρμόζονται.
          Αντίθετα, ένα άκρη-προς-άκρη (\en{end-to-end}) σύστημα θα απαιτούσε τη συλλογή μεγάλου πλήθους δεδομένων εξειδικευμένων στο συγκεκριμένο πρόβλημα της διπλωματικής.

          Επίσης, αυτή η προσέγγιση επιτρέπει τον ευκολότερο εντοπισμό των πηγών του λάθος αποτελέσματος όπως φαίνεται και στην \fref{sec:problems}.
    \item Τον βαθμό αποτελεσματικότητας μεθόδων μηχανικής μάθησης εκπαιδευμένων σε σχετικά μικρό αριθμό δεδομένων.
          Η δημιουργία του συνόλου εκπαίδευσης πραγματοποιήθηκε με περιορισμένους πόρους και εμπειρία στη συλλογή δεδομένων.
    \item Έναν καινοτόμο τρόπο διαχωρισμού πολλαπλών προθέσεων ανά πρόταση.
          Υποστηρίζονται δύσκολες δομές όπως αυτή της ανύψωσης δεξιού κόμβου.
          Δεν απαιτείται η συλλογή δεδομένων με παραδείγματα προτάσεων πολλαπλών προθέσεων.
          Αυτά θα πλήθυναν σημαντικά τον απαιτούμενο αριθμό δεδομένων και θα απαιτούσαν μια πολυωνυμική αύξηση τους με κάθε νέα πρόθεση στο σύστημα.
    \item Ένα σύστημα που μπορεί να αποδειχθεί χρήσιμο εργαλείο σε έναν χρήστη εφόσον μάθει τους περιορισμούς του και τον τρόπο σύνταξης που οδηγεί σε βέλτιστα αποτελέσματα.
          Αυτή η διαδικασία μάθησης θεωρείται ότι δεν απαιτεί τεχνικές και εσωτερικές γνώσεις πάνω στο \projectname{}.
\end{itemize}

Περιορισμοί του συστήματος και πιθανά ελαττώματα είναι:
\begin{itemize}
    \item Η τμηματοποίηση σε μονάδες μπορεί να οδηγεί σε απώλεια πληροφορίας.
          Δηλαδή, η εκπαίδευση του μοντέλων ενός υποπροβλήματος μπορεί να επωφελούνταν από τις πληροφορίες που συνδέεται με κάποιο άλλο υποπρόβλημα λόγω της μεταξύ τους στατιστικής συσχέτισης.
          Για παράδειγμα, ένα κοινό μοντέλο για τον διαχωρισμό και αναγνώριση προθέσεων μπορεί να λάμβανε υπόψη λεπτομέρειες σχετικές με τον συγκεκριμένο συνδυασμό δύο προθέσεων σε μια πρόταση.
    \item Τα μοντέλα που επιλύουν γενικά προβλήματα, αν και παρουσιάζουν καλύτερες επιδόσεις στη γενική περίπτωση, μπορεί να εμφανίζουν ελλείψεις λόγω διαφοράς λεξικολογίου στον συγκεκριμένο τομέα του προβλήματος.
    \item Δεν παρουσιάζει καλή απόδοση σε κείμενα που απαιτούν υψηλότερα επίπεδα λογικής όπως για παράδειγμα σε πρότασης που εμφανίζονται σε δοκιμασίες \lib{Winograd}~\cite{levesque2012winograd}.
    \item Απαιτείται η είσοδος γραμματικά, συντακτικά και ορθογραφικά ορθού κειμένου με τις παραθέσεις να περιλαμβάνονται πάντα μέσα σε εισαγωγικά.
          Αυτή η έλλειψη ευρωστίας μπορεί να δυσχεραίνει τη σύνδεση του \projectname{} με την έξοδο με ένα σύστημα \newterm{Αυτόματη\dd{ς} Αναγνώριση\dd{ς} Ομιλίας}{Automatic Speech recognition}.
    \item Κατά την αναγνώριση πρόθεσης δεν γίνεται εκμετάλλευση των συμφραζομένων των προηγούμενων προτάσεων.
          Ο μόνος τρόπος που οι περασμένες προτάσεις επηρεάζουν την επεξεργασία της τρέχουσας πρότασης είναι μέσω συναναφορών.
\end{itemize}

\section{Μελλοντική εργασία}
Είναι δυνατή η αντιμετώπιση μερικών περιορισμών του συστήματος με μελλοντικές επεκτάσεις.
Κάποιες βασικές ιδέες είναι:
\begin{itemize}
    \item Χρήση ρομποτικών οντολογιών.
          Οι \newterm{Οντολογί\rr{ες}{α}}{Ontology} παρέχουν ένα δομημένο τρόπο οργάνωσης πληροφοριών,
          υποστηρίζουν μεθόδους για την ανάκτηση αποθηκευμένων δεδομένων μέσω \newtermprint[Query]{ερωτημάτων}
          και επιτρέπουν την ανάλυση των σχέσεων μεταξύ δεδομένων~\cite{diamantopoulos2017software}.

          % TODO: 3.x and just link
          Αυτές επιτρέπουν την ενσωμάτωση πληροφοριών του εξωτερικού κόσμου και μπορούν να βοηθήσει στη καλύτερη κατανόηση των προθέσεων του χρήστη.
          Για παράδειγμα, οι \citet{hu2009understanding} χρησιμοποιείται η \en{Wikipedia} ως οντολογία σε συνδυασμό με τυχαίους περιπάτους για την αναγνώριση της πρόθεσης του χρήστη.
          Στο~\cite{zang2018translating} οι εντολές του χρήστη συνδέονται με έναν χάρτη του περιβάλλοντος.
          Στα~\cite{salamknowledge,zhang2018learning,nyga2018grounding} οι οντολογίες χρησιμοποιούνται έτσι ώστε το ρομπότ να έχει καλύτερη κατανόηση των οντοτήτων που περιέχονται στα λεγόμενα του χρήστη.
          Τέλος, οι \citet{diamantopoulos2017software} χρησιμοποιούν οντολογία για την ανάλυση λειτουργικών απαιτήσεων.

          Η υλοποίηση μιας οντολογίας που μπορεί να καλύψει επαρκώς τη χρήση του \metamodel{} θεωρήθηκε εκτός του πλαισίου αυτής της διπλωματικής.
          Επίσης, μια οντολογία που συνδέεται με το τρέχων περιβάλλον του ρομπότ (όπως για παράδειγμα το \libcite{KnowRob} προσφέρει πληροφορίες που δεν γενικεύονται για κάθε ρομποτική εφαρμογή.
    \item Βελτίωση της ανίχνευσης φυσικών μονάδων απόστασης και ταχύτητας που χρησιμοποιούνται για τη πρόθεση \intent{BodyMotion}.
          Αυτό μπορεί να γίνει με υλοποιήσεις επεκτάσεων στα \libcite[Python!]{Duckling} ή \libcite{Rustling} και τη καλύτερη ενσωμάτωσή τους στη τρέχουσα σωλήνωση λογισμικού.
    \item Αναγνώριση προθέσεων που λαμβάνει υπόψη το ιστορικό των προηγούμενων προτάσεων.
          Ιδέες μπορούν να δανειστούν από τα συστήματα διαλόγων~\cite{bhargava2013easy} ή από άλλες δημοσιεύσεις ταξινόμησης κειμένου~\cite{lee2016sequential}.
          Αυτός ο στόχος είναι δύσκολο να εκπληρωθεί στο πλαίσιο αυτής της διπλωματικής λόγω της ανάγκης ύπαρξης μεγάλου αριθμού δεδομένων με ολοκληρωμένα σενάρια χρήσης όπως αυτά που εμφανίζονται στο \fref{chap:results}.

          Επιπρόσθετα, ίσως είναι εφικτή και η ενίσχυση του μοντέλου αναγνώρισης πρόθεσης με παρόμοια χαρακτηριστικά.
    \item Εκμετάλλευση του αποτελέσματος της ανάθεσης σημασιολογικών ρόλων για τη βελτίωση της απόδοσης πλήρωσης υποδοχών στη μονάδα \NLU{}.
          Μπορούν να δημιουργηθούν συναρτήσεις χαρακτηριστικών για τα \CRFR{} που το αποτέλεσμά τους να εξαρτάται από τον σημασιολογικό ρόλο κάθε λέξης.
    \item Αυτόματη αναγνώριση παραθέσεων που δε βρίσκονται ανάμεσα σε εισαγωγικά και πλάγιου λόγου όπως, για παράδειγμα, στο~\cite{pareti2013automatically}.
    \item Επανεκπαίδευση του \libcite[Python!]{spaCy} σε προτάσεις που παρουσιάζει πρόβλημα στην επισημείωση μερών του λόγου (βλέπε και \fref{subsec:problems-spacy}).
          Αυτό θα βοηθήσει στη καλύτερη απόδοση του μοντέλου στο συγκεκριμένο λεξιλόγιο που συναντάται στα σενάρια χρήσης του \projectname{}.
    \item Αξιοποίηση μεταφοράς μάθησης μέσω \libcite{BERT}.
          Αυτό μπορεί να οδηγήσει σε βελτίωση της απόδοσης των μοντέλων του \libcite[Python!]{AllenNLP}\footnote{\url{https://github.com/allenai/allennlp/pull/2854}}.
          Μπορεί να διερευνηθεί και η αξιοποίησή τους στο πρόβλημα της επίλυσης σχέσεων συναναφοράς.

          Επίσης, είναι πιθανή η η εξαγωγή \newterm{\rr{Διανυσμάτων}{Διανύσματα} Λέξεων}{Word Vectors} και η σύγκρισή τους με αυτά των \libcite{GloVe} και \libcite{ELMO}.

          Αναφέρεται ότι διερευνήθηκε η χρήση ταξινόμησης μέσω νευρωνικών δικτύων βασισμένα στο \lib{BERT} αλλά τελικά απορρίφθηκε καθώς αποτελεί υπολογιστικά χρονοβόρα διαδικασία
          ενώ η τρέχουσα μονάδα \NLU{} βασίζεται στην αξιολόγηση πολλών συνδυασμών ανά πρόταση.
          Για να προσπεραστεί αυτό το πρόβλημα, γίνεται να επιχειρηθεί η εκπαίδευση μοντέλων ταξινόμησης ακολουθιών πολλών ετικετών βασισμένο και πάλι στο \lib{BERT} που θα αντικαταστήσει ολικά τη μονάδα NLU.
          Μια άλλη πιθανή προσέγγιση εδώ είναι της \newterm{Μάθησης Πολλών Στόχων}{Multi-Task Learning} κατά την οποία συνήθως επιχειρείται η ταυτόχρονη επίλυση πολλών προβλημάτων,
          δηλαδή η βελτιστοποίηση γίνεται πάνω σε πολλές συναρτήσεις απωλειών.
\end{itemize}

% vim:ts=4:sw=4:expandtab:fo-=tc:tw=120
