\chapter{Εισαγωγή}\label{chap:intro}

\section{Κίνητρο}\label{sec:motivation}
Από τα αρχαία χρόνια, ο άνθρωπος ονειρευόταν μηχανές που μιμούνται οργανισμούς ή ακόμα και ξεπερνούν τους ανθρώπους στις ικανότητές τους.
Μια από τις αρχαιότερες αναφορές σε αυτόνομα, αυτοκινούμενα ρομπότ πρέπει να είναι αυτή στην Ιλιάδα του Όμηρου, γύρω στο 800 π.Χ.
Εκεί, αναφέρεται πως ο Ήφαιστος κατασκεύασε τρίποδες \enquote{με ρόδες χρυσές για να μπορούν αυτόματα να μπαίνουν στων θεών τη σύναξη και πάλι μόνοι τους να γυρνούν στο οίκημα}.
Οι τρίποδες αυτοί περιγράφονται ως ισχυροί και έξυπνοι, με αυτιά και φωνές, πρόθυμοι να βοηθήσουν και να εργαστούν~\cite{graefe2009ancient}.

Η λέξη \emph{ρομπότ} προέρχεται από την τσεχική λέξη \enquote{robota} που σημαίνει δουλειά, εργασία.
Επινοήθηκε από τον αδερφό του Τσέχου θεατρικού συγγραφέα Κάρελ Τσάπεκ (Karel Čapek) το 1920.
Η λέξη έχει καταλήξει να περιλαμβάνει κάθε μηχανική συσκευή που μπορεί να εκτελέσει εργασίες οι οποίες κατά κύριο λόγο εκτελούνται από ανθρώπους.
Συνήθως, θεωρείται ότι η συσκευή αυτή έχει ανθρωποειδή χαρακτηριστικά, αν και αυτό δεν είναι αναγκαστικό~\cite{asimov1989asimov}.

Ο κλάδος της ρομποτικής ασχολείται με τη σύνθεση διάφορων ανθρώπινων λειτουργιών με τη χρήση ποικίλων μηχανισμών, αισθητήρων, ενεργοποιητών και υπολογιστών.
Η σύνθεση αυτή αποτελεί τεράστιο εγχείρημα και απαιτεί τον δανεισμό πληθώρας ιδεών από πολλούς \enquote{κλασικούς} επιστημονικούς κλάδους~\cite{craig2009introduction}.

Καθώς η τεχνολογία που χρησιμοποιούμε καθημερινά γίνεται ολοένα και πιο περίπλοκη, προσπαθούμε να βρίσκουμε τρόπους που καθιστούν οικεία και φιλική την αλληλεπίδραση με αυτή.
Οι άνθρωποι και άλλα πρωτεύοντα θηλαστικά έχουν εξελιχθεί ώστε να έχουν εγκέφαλους ικανούς για άριστη κοινωνική αλληλεπίδραση,
οπότε η προσπάθεια χρήσης της φυσικής γλώσσας για την επικοινωνία με μηχανήματα είναι ένας λογικά επόμενος στόχος~\cite{dunbar2007evolution,breazeal2004designing}.
Ακόμα και στην αρχαία μυθολογία, ο Ήφαιστος έδωσε ανθρώπινη φωνή στις χρυσές μηχανικές υπηρέτριές του, καθιστώντας τες πιο αποτελεσματικά εργαλεία~\cite{gera2003ancient}.

Τα \enquote{παραδοσιακά} συστήματα ρομπότ συνήθως σχεδιάζονται από επαγγελματίες μηχανικούς και ο προγραμματισμός τους γίνεται επίσης από ειδικούς και αφορά σε ένα συγκεκριμένο σκοπό.
Για παράδειγμα, ένα βιομηχανικό ρομπότ απαιτεί την εργασία ηλεκτρολόγων και μηχανολόγων μηχανικών καθώς και διάφορων μηχανικών λογισμικού για να επιτευχθεί η άρτια συνεργασία των αισθητήρων, των μηχανολογικών μερών και του λογισμικού.
Για αυτό τον λόγο, η δυνατότητα προγραμματισμού, σε υψηλό τουλάχιστον επίπεδο, ρομποτικών εφαρμογών μπορεί να βοηθήσει στη σημαντική μείωση του κόστους ανάπτυξης και να επιτρέψει σε άτομα χωρίς επαγγελματική εμπειρία να χρησιμοποιήσουν προγραμματιζόμενα ρομποτικά συστήματα γενικής χρήσης όπως το NAO\footnote{\url{https://www.softbankrobotics.com/emea/en/nao}}.

\section{Περιγραφή του προβλήματος}\label{sec:problem-description}
Το γενικό πρόβλημα που τίθεται προς λύση είναι η αυτόματην παραγωγή μιας ρομποτικής εφαρμογής με βάση την περιγραφή της από τον χρήστη σε φυσική γλώσσα.
Ένα κατάλληλα εξοπλισμένο ρομπότ θα μπορεί να εκτελέσει τον παραγόμενο κώδικα, μεταφέροντας τη διαδικασία που περιγράφεται στον φυσικό κόσμο.

Πιο συγκεκριμένα, γίνεται προσπάθεια αντιστοίχησης ενός κειμένου αγγλικής γλώσσας σε ένα σύνολο προκαθορισμένων ενεργειών που προσφέρονται από μια ανεξάρτητη πλατφόρμα λογισμικού.
Η παρούσα διπλωματική ασχολείται με το κομμάτι της στατικής αντιστοίχησης των προτάσεων του κειμένου στις κατάλληλες ενέργειες.
Δεν γίνεται προσπάθεια συνδυασμού τους για την παραγωγή ενός ολοκληρωμένου αλγορίθμου που υλοποιεί τη λογική που προδιαγράφεται στο κείμενο,
το πρόβλημα αυτό θεωρείται εκτός του πεδίου μελέτης και μπορεί να επιλυθεί ξεχωριστά αξιοποιώντας τα αποτελέσματα αυτής της εργασίας.

\section{Στόχοι της διπλωματικής}\label{sec:diploma-purpose}
Στόχος της συγκεκριμένης διπλωματικής εργασίας είναι η επεξεργασία γραπτών προτάσεων μέσω τεχνικών
\newterm{Επεξεργασία\dd{ς} Φυσικής Γλώσσας}{Natural Language Processing - NLP}.
Οι προτάσεις αυτές περιγράφουν μια ρομποτική διαδικασία που μπορεί να εκτελεστεί από το δοθέν ρομπότ.

Η ρομποτική αυτή διαδικασία αποτελείται από προκαθορισμένες ενέργειες που περιέχονται στο μετα-μοντέλο \metamodel{}\footnote{\url{https://r4a.issel.ee.auth.gr/nao4a/}}.
Σε κάθε πρόταση γίνεται προσπάθεια αναγνώρισης των \newterm{\rr{προθέσεων}{Αναγνώριση Πρόθεσης}}{Intent\rr{}{ Identification}} του χρήστη,
που θεωρείται ότι αντιστοιχούν σε κάποιες από τις προκαθορισμένες ενέργειες του μέτα-μοντέλου.

Δεν γίνεται προσπάθεια επίλυσης ορθογραφικών, συντακτικών και γραμματικών λαθών στο κείμενο εισόδου.

\section{Διάρθρωση εγγράφου}\label{sec:structure}
Η αναφορά της παρούσας διπλωματικής εργασίας περιλαμβάνει έξι κεφάλαια:
\begin{compactenum}
    \item Το \hyperref[chap:intro]{τρέχον} αποτελεί την εισαγωγή.
    \item Το \hyperref[chap:background]{δεύτερο} παρουσιάζει το σχετικό υπόβαθρο\anoteleia{}
          βασικές έννοιες που χρησιμοποιήθηκαν καθώς και τους αλγορίθμους μηχανικής μάθησης στους οποίους βασίζεται η υλοποίηση της διπλωματικής.
    \item Το \hyperref[chap:state-of-the-art]{τρίτο} συνοψίζει την ερευνητική περιοχή σχετική με το αντικείμενο της εργασίας.
    \item Το \hyperref[chap:methodology]{τέταρτο} αποτελείται από την επεξήγηση της μεθοδολογίας που ακολουθήθηκε και την λειτουργία της σωλήνωσης λογισμικού που υλοποιήθηκε.
    \item Το \hyperref[chap:results]{πέμπτο} παρουσιάζει τα αποτελέσματα του \projectname{} σε διάφορα σενάρια χρήσης.
    \item Το \hyperref[chap:conclusions_future]{έκτο} καταλήγει με τα συμπεράσματα και τις πιθανές μελλοντικές επεκτάσεις.
\end{compactenum}

% vim:ts=4:sw=4:expandtab:fo-=tc:tw=120
