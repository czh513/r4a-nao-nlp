\begin{center}
    \vspace{0.3cm}
    \textbf{\Large{Περίληψη}}
    \phantomsection
    \addcontentsline{toc}{section}{Περίληψη}

    \vspace{0.1cm}
\end{center}

Η ραγδαία πρόοδος της τεχνολογίας τις τελευταίες δεκαετίες χαρακτηρίζεται και από μια ανάλογη αύξηση στην πολυπλοκότητά της,
αναγκάζοντάς μας να αναζητούμε τρόπους που καθιστούν την αλληλεπίδρασή μας με αυτή οικεία και φιλική.
Η πρόσφατη κατακόρυφη άνοδος της δημοτικότητας και αποτελεσματικότητας των μεθόδων μηχανικής μάθησης
και του κλάδου της τεχνητής νοημοσύνης κατ' επέκταση
έχει οδηγήσει στην προσδοκία της φυσικής επικοινωνίας του ανθρώπου με τα ρομπότ.
Ως εκ τούτου, η επικοινωνία μέσω φυσικής γλώσσας με αυτά είναι ένας λογικά επόμενος στόχος και για τον λόγο αυτό έχουν αναπτυχθεί πρόσφατα διάφορα συστήματα κατανόησης φυσικής γλώσσας.

Προγραμματιζόμενα ρομποτικά συστήματα γενικής χρήσης όπως το NAO μπορούν να αξιοποιηθούν για την καθημερινή προσωπική χρήση από τον μέσο άνθρωπο.
Η δυνατότητα προγραμματισμού τους από ανθρώπους χωρίς ιδιαίτερη τεχνική γνώση μέσω της χρήσης φυσικής γλώσσας μπορεί να οδηγήσει σε σημαντική βελτίωση της χρησιμότητάς τους.

Στόχος της παρούσας διπλωματικής είναι η αναγνώριση ενεργειών μέσα σε ένα κείμενο φυσικής γλώσσας και η αντιστοίχισή τους σε μια ήδη υπάρχουσα ρομποτική πλατφόρμα.
Δεν επιχειρείται ο συνδυασμός αυτών των ενεργειών για την παραγωγή ενός αλγορίθμου που να ακολουθεί τη λογική που εκφράζει το κείμενο
αλλά γίνεται απλώς στατική αντιστοίχιση των προτάσεων του κειμένου στις κατάλληλες ενέργειες.
Η έξοδος του συστήματος που υλοποιείται μπορεί να αξιοποιηθεί από κάποια άλλη εφαρμογή για την τελική παραγωγή εκτελέσιμου κώδικα.

Για τους προαναφερθέντες στόχους, αναπτύξαμε ένα σύστημα κατανόησης φυσικής γλώσσας (NLU), το \projectname{}, το οποίο αναγνωρίζει τις ενέργειες που υποστηρίζονται από το μετά-μοντέλο \metamodel{}.
Εφαρμόζουμε μια σωλήνωση λογισμικού που τμηματίζει το κείμενο χρησιμοποιώντας την ανάθεση σημασιολογικών όρων για να ταυτοποιήσει πολλαπλές προθέσεις του χρήστη ανά πρόταση.
Επιπλέον, το σύστημα αξιοποιεί τα αποτελέσματα της επίλυσης συναναφοράς σε όλο το κείμενο προκειμένου να βελτιώσει την απόδοση της ταξινόμησης προθέσεων και πλήρωσης υποδοχέων σε προτάσεις που περιλαμβάνουν αναφορές.
Καθώς το σύνολο δεδομένων για την εκπαίδευση του συστήματος NLU έπρεπε να δημιουργηθεί εκ του μηδενός, η προσέγγισή μας σχεδιάστηκε έτσι ώστε να ανταποκρίνεται σε χαμηλό αριθμό δεδομένων.
Δεν υπάρχουν απαιτήσεις στο σύνολο εκπαίδευσης για προτάσεις που συνδυάζουν πολλαπλές προθέσεις, μιας και κάτι τέτοιο θα οδηγούσε σε πολυωνυμική αύξηση του μεγέθους του.
Η έξοδος της σωλήνωσης είναι ένας κατευθυνόμενος γράφος που περιέχει όλες τις ανιχνευμένες ενέργειες και τις συνδέει χρησιμοποιώντας τους αρχικούς συνδέσμους του κειμένου.

Η υλοποίηση αυτή επωφελείται από τον διαχωρισμό σε υποπροβλήματα καθώς τα μοντέλα που χρησιμοποιούνται, εξαιρουμένων αυτών που εκτελούν ταξινόμηση προθέσεων και πλήρωση υποδοχέων,
είναι προ-εκπαιδευμένα σε μεγάλα σύνολα δεδομένων και αφορούν μείζονες εργασίες στον τομέα της επεξεργασίας φυσικής γλώσσας και ως εκ τούτου, προβλέπεται να βελτιωθούν με την περαιτέρω ανάπτυξη της σχετικής τεχνολογίας.
Θεωρείται ότι η παρούσα προσέγγιση μπορεί να αξιοποιηθεί, χωρίς να υπάρχει η ανάγκη να αυξηθούν τα δεδομένα εκπαίδευσής τους, από συστήματα διαλόγου προσανατολισμένα για καθήκοντα ή άλλες σχετικές εφαρμογές που συχνά δεν έχουν την ικανότητα να αναγνωρίζουν πολλαπλές προθέσεις ανά πρόταση.

Εν κατακλείδι, αναπτύχθηκε ένα σύστημα που μπορεί να αποδειχτεί χρήσιμο στον τελικό χρήστη ο οποίος μπορεί να αποκτήσει βέλτιστα αποτελέσματα αν γνωρίσει τους περιορισμούς και τις ιδιαιτερότητες.
Αυτή η διαδικασία δεν θεωρείται ότι απαιτεί τεχνικές ή εσωτερικές γνώσεις πάνω στο \projectname{}.

% vim:ts=4:sw=4:expandtab:fo-=tc:tw=120
