\section{Κατανόηση φυσικής γλώσσας}
\subsection{Όροι γλωσσολογίας}\label{subsec:linguistics}
Οι παρακάτω όροι χρησιμοποιούνται στο πλαίσιο της παρούσας διπλωματικής.
Η λίστα δεν είναι εξαντλητική απλώς περιλαμβάνει όρους που χρειάζονται αποσαφήνιση.
\begin{itemize}
    \item \newtermprintc{Πρόταση} (\en{Sentence}, \en{Clause} και \en{Proposition}):
          H απόδοση στα ελληνικά των τριών αυτών όρων είναι προβληματική.
          Και οι τρεις μεταφράζονται ως \enquote{πρόταση} αλλά υπάρχουν οι εξής διαφορές:
          \begin{compactitem}
              \item Ο όρος \en{clause} αποδίδεται ως \newtermprint{απλή πρόταση}\index{Απλή Πρόταση - Clause}, π.χ.\ \engquote{The sun is shining}.
              \item Ο όρος \en{sentence} αποδίδεται ως \newtermprint{πρόταση} αλλά σημαίνει \newtermprint{γραμματική περίοδο}\index{Γραμματική Περίοδος - Sentence},
                    η οποία είναι δυνατό να αποτελείται από περισσότερες επιμέρους προτάσεις ή κώλα (νοηματικές ενότητες),
                    π.χ.\ \engquote{The sun shines when there are no clouds in the sky}.
              \item Ο όρος \en{proposition} αποτελεί έννοια του σημασιολογικού επιπέδου,
                    μιας και δεν αναφέρεται στη γραμματική αλλά στη \newtermprint{λογική πρόταση}\index{Λογική Πρόταση - Preposition}.
                    Για παράδειγμα, οι δομές \engquote{Peter solved the exercise} και \engquote{The exercise was solved by Peter} αποτελούν διαφορετικές γραμματικές προτάσεις αλλά είναι εκδοχές της ίδιας λογικής πρότασης,
                    εφόσον έχουν κοινό λογικό υποκείμενο, τον \engquote{Peter}, και κοινό κατηγόρημα, την επίλυση της άσκησης.
          \end{compactitem}
          Μια πρόταση τυπικά περιλαμβάνει ένα \newterm{Υποκείμενο}{Subject} και ένα \newterm{Κατηγόρημα}{Predicate}.
          Ο όρος κατηγόρημα κατά κύριο λόγο αναφέρεται στο ρήμα της πρότασης και σε εκείνες τις λέξεις που το συνοδεύουν.
    \item \newtermc{Επίρρημα}{Adverb} (από ad (\engquote{to}) + verbum (\engquote{word})):
          λέξη που προσδιορίζει ρήματα, προτάσεις, επίθετα και άλλα επιρρήματα.
          Παραδείγματα:
          \begin{compactitem}
              \item \engquote{\textit{I work often}}: το \engquote{often} προσδιορίζει το ρήμα \engquote{work}.
              \item \engquote{\textit{She was very pretty}}: το \engquote{very} προσδιορίζει το επίθετο \engquote{pretty}.
              \item \engquote{\textit{Often, there have been unacceptable delays}}: το \engquote{often} προσδιορίζει την υπόλοιπη πρόταση.
          \end{compactitem}
    \item \newtermc{Σύνδεσμος}{Conjunction}:
          μέρος του λόγου που χρησιμοποιείται για να συνδέσει λέξεις, φράσεις ή απλές προτάσεις (\en{Clause}).
          Για παράδειγμα, \engquote{I was tired, so I went to bed}: ο σύνδεσμος \engquote{so} ενώνει τις δύο απλές προτάσεις.
    \item \newtermc{Παράταξη}{Coordination}:
          Στην παρατακτική σύνδεση οι προτάσεις παρατάσσονται, δηλαδή μπαίνουν η μία δίπλα στην άλλη και συνδέονται μεταξύ τους με συνδέσμους,
          (συμπλεκτικούς, αντιθετικούς, διαχωριστικούς, συμπερασματικούς κτλ.) που ονομάζονται παρατακτικοί σύνδεσμοι.
    \item \newtermc{Ανύψωση Δεξιού Κόμβου}{Right Node Raising}:
          Ο όρος υποδηλώνει έναν μηχανισμό μοιράσματος με συμμετρικά χαρακτηριστικά,
          σύμφωνα με τον οποίο το υλικό στο δεξί άκρο μιας παράταξης να είναι κατά κάποιον τρόπο κοινό και να ερμηνεύεται ως τμήμα και των δύο συστατικών που συμπλέκονται παρατακτικά.
          Οι παρατάξεις συστηματικά επιτρέπουν μία ακολουθία να προφέρεται μία μόνο φορά, στο δεξί άκρο της παράταξης.
          Για παράδειγμα, \engquote{[Sam likes] but [Fred dislikes] feta cheese}.
    \item \newtermc{Τροποποιητής}{Modifier}\footnote{\url{http://www.greek-language.gr/greekLang/modern_greek/tools/lexica/glossology/show.html?id=333}}~\cite{komvos}:
          Κάθε ονοματική ή ρηματική φράση μπορεί να τροποποιηθεί σημασιολογικά με την χρήση ενός τροποποιητή.
          Οι τροποποιητές είναι προσδιοριστές, δηλαδή συστατικά που περιορίζουν το σημασιακό πεδίο μιας ονοματικής ή ρηματικής φράσης.
          Τα επίθετα, οι μετοχές, οι αναφορικές προτάσεις ή ορισμένες προθετικές φράσεις λειτουργούν ως τροποποιητές ονοματικών φράσεων, ενώ τα επιρρήματα και οι επιρρηματικές δευτερεύουσες προτάσεις είναι τροποποιητές ρηματικών φράσεων.
    \item \newtermc{Αμφισημία}{Ambiguity}~\cite{lexica-glossology}\footnote{\url{http://www.greek-language.gr/greekLang/modern_greek/tools/lexica/glossology/show.html?id=372}}:
          Ο όρος αμφισημία αναφέρεται στην πολλαπλότητα των σημασιών η οποία είναι δυνατό να χαρακτηρίζει μια λέξη ή μια πρόταση.
          Στον πραγματωμένο λόγο, δηλαδή στα φυσικά εκφωνήματα, αίρεται σχεδόν πάντα από τα συμφραζόμενα σε μια δεδομένη περίσταση επικοινωνίας.
          Η συντακτική ή δομική αμφισημία προκύπτει όταν μια πρόταση είναι δυνατόν να αναλύεται από συντακτική άποψη με περισσότερους από έναν τρόπους.
          Για παράδειγμα, η πρόταση \engquote{I saw a girl with a telescope} μπορεί να σημαίνει
          ότι εγώ χρησιμοποίησα τηλεσκόπιο για να δω ένα κορίτσι ή ότι είδα ένα κορίτσι που είχε ένα τηλεσκόπιο.

          Η τέλεια κατανόηση της γλώσσας από έναν υπολογιστή θα απαιτούσε ένα σύστημα τεχνητής νοημοσύνης που θα είχε τη δυνατότητα να επεξεργαστεί όλη την ανθρώπινη γνώση,
          γεγονός που με τη σειρά του θα απαιτούσε ένα σύστημα γενικής τεχνητής νοημοσύνης.
\end{itemize}

\subsection{Επισημείωση μερών του λόγου}
Η \newterm{Επισημείωση Μερών του Λόγου}{Part-of-Speech Tagging} αποτελεί μια βασική διεργασία της επεξεργασίας φυσικής γλώσσας.
Πρόκειται για τη διαδικασία αντιστοίχισης μιας λέξης σε ένα κομμάτι κειμένου με το αντίστοιχο μέρος του λόγου.

Συνήθως χρησιμοποιούνται μέθοδοι επισημείωσης ακολουθιών, όπως τα \CRFR{}.
Η είσοδος $\vx$ που αποτελείται από $T$ λέξεις διαιρείται στα διανύσματα χαρακτηριστικών $\{\vx_0,\allowbreak \vx_1,\allowbreak \ldots,\allowbreak \vx_T\}$
και η έξοδος $y_t$ για την $t$-οστή λέξη αποτελεί το αντίστοιχο μέρος του λόγου.
Κάθε $\vx_t$ μπορεί να περιέχει διάφορες πληροφορίες σχετικά με τη λέξη στη θέση $t$,
όπως την ταυτότητά της, ορθογραφικά χαρακτηριστικά όπως προθέματα και επιθέματα, πληροφορίες σε σημασιολογικές βάσεις δεδομένων όπως το \libcite{WordNet} και το λήμμα της σε ειδικά λεξιλόγια~\cite{sutton2012introduction}.

\subsection{Σημασιολογικοί ρόλοι}\label{subsec:srl}
Η \newterm{Ανάθεση Σημασιολογικών Ρόλων}{Semantic Role Labeling - SRL} στοχεύει να ανακτήσει τη \newterm[Ανάθεση Σημασιολογικών Ρόλων - Semantic Role Labeling - SRL]{Δομή Κατηγορήματος-Ορισμάτων}{Predicate-Argument Structure} μιας πρότασης.
Το \newterm{Κατηγόρημα}{Predicate} μιας πρότασης (τυπικά ένα ρήμα) δηλώνει \enquote{τι} έλαβε χώρα και τα άλλα μέρη της πρότασης δηλώνουν τους συμμετέχοντες στο γεγονός (όπως \enquote{ποιος} και \enquote{πού}),
καθώς και άλλα δεδομένα (όπως \enquote{πότε} και \enquote{πώς})~\cite{marquez2008semantic}.
\newtermc[Ανάθεση Σημασιολογικών Ρόλων - Semantic Role Labeling - SRL]{\rr{Όρισμα}{Ορίσματα}}{Argument\dd{s}} είναι μια έκφραση που βοηθάει στην ολοκλήρωση του νοήματος του κατηγορήματος.

Το πρωταρχικό καθήκον της ανάθεσης σημασιολογικών ρόλων είναι να υποδείξει ποιες σημασιολογικές σχέσεις επικρατούν ανάμεσα σε ένα κατηγόρημα και τα ορίσματά του.
Αυτές οι σχέσεις προέρχονται από έναν προκαθορισμένο κατάλογο πιθανών σημασιολογικών ρόλων για αυτό το κατηγόρημα (ή ομάδα κατηγορημάτων).
Για να επιτευχθεί αυτό, τα συστατικά μιας πρότασης που είναι φορείς ρόλου πρέπει να ταυτοποιηθούν και να τους αποδοθούν οι σωστοί σημασιολογικοί ρόλοι, όπως για παράδειγμα:
\engquote{[The girl on the swing]\textsubscript{Δράστης (Agent)} [whispered]\textsubscript{Κατηγόρημα (Predicate)} to [the boy beside her]\textsubscript{Λήπτης (Recipient)}}.

Ανάθεση σημασιολογικών ρόλων έχει πραγματοποιηθεί σε μεσαία έως μεγάλα σώματα κειμένου στα \libcite{FrameNet} και \libcite{PropBank}, επιτρέποντας έτσι την ανάπτυξη προσεγγίσεων μηχανικής μάθησης.
Το \lib{PropBank} προσθέτει ένα στρώμα σημασιολογικής επισηµείωσης~\cite{kingsbury2002adding} στις συντακτικές δοµές του \lib{Penn Treebank}~\cite{penntreebank}.
% Εκτός από τον σχολιασμό των σημασιολογικών ρόλων, ο σχολιασμός της \lib{PropBank} απαιτεί την επιλογή μιας ταυτότητας (id) νοήματος (γνωστή και ως id frameset ή roleset) για κάθε κατηγόρημα.
Αυτό επιτρέπει επίπεδα σημασιολογικής ανάλυσης που δεν είναι εφικτά μόνο με συντακτική ανάλυση.
Για παράδειγμα, ενώ στις προτάσεις \engquote{John broke the window} και \engquote{The window broke}
η λέξη \engquote{window} είναι το αντικείμενο της πρώτης και στη δεύτερη είναι το υποκείμενο, ο σημασιολογικός του ρόλος είναι ίδιος και στις δύο περιπτώσεις.

Συγκεκριμένα, στο \lib{PropBank}, σε κάθε ρήμα αντιστοιχεί ένα σύνολο σημασιολογικών πλαισίων (αναφέρονται ως \en{Frameset}).
Ο διαχωρισμός γίνεται σύμφωνα με τη σημασία (ρόλο) του ρήματος ανά περίπτωση\anoteleia{} συντακτικές τροποποιήσεις που διατηρούν τη σημασία θεωρούνται ότι ανήκουν στο ίδιο σύνολο.
Κάθε πλαίσιο δέχεται μέχρι και 6 ορίσματα \ARGs{} στα οποία δίνεται μια περιγραφή του ρόλου τους σε σχέση με το κατηγόρημα.
Επίσης οι \newterm{Τροποποιητ\rr{ές}{ής}}{Modifier} ή \newterm{Προσδιορισμ\rr{οί}{ος}}{Adjunct} ονοματίζονται σύμφωνα με τον ρόλο τους,
όπως \engquote{TMP} (\en{Temporal} - Χρονικός) και \engquote{LOC} (\en{Locative} - Χωρικός),
και εφαρμόζονται ισάξια σε κάθε πλαίσιο κάθε ρήματος.
Για παράδειγμα, η επισημείωση της πρότασης \engquote{Mr. Bush met him privately, in the White House, on Thursday} προκύπτει~\cite{propbank2015guidelines}:
\begin{compactitem}
    \item κατηγόρημα: \engquote{met}\footnote{\url{http://verbs.colorado.edu/propbank/framesets-english-aliases/meet.html}}
    \item \enttt{ARG0}: \engquote{Mr. Bush} --- δράστης
    \item \enttt{ARG1}: \engquote{him} --- το πρόσωπο που συναντάει ο δράστης
    \item \enttt{ARGM-MNR}: \engquote{privately} --- ο τρόπος συνάντησης
    \item \enttt{ARGM-LOC}: \engquote{in the White House} --- το μέρος συνάντησης
    \item \enttt{ARGM-TMP}: \engquote{on Thursday} --- ο χρόνος συνάντησης
\end{compactitem}

Η ανάθεση σημασιολογικών ρόλων αποτελεί σημαντικό βήμα για την κατανόηση της σημασίας μιας πρότασης και έχει δειχθεί ότι μπορεί να βελτιώσει αποτελέσματα σε άλλα προβλήματα επεξεργασίας φυσικής γλώσσας
όπως τη \newtermprint[Machine Reading]{Μηχανική Ανάγνωση}~\cite{berant2014modeling,wang2015machine},
την \newtermprint[Question Answering]{Απάντηση Ερωτήσεων}~\cite{narayanan2004question,weston2015towards,yih2016value},
τη \newtermprint[Machine Translation]{Μηχανική Μετάφραση}~\cite{wu2009semantic,liu2010semantic,gao2011utilizing,bazrafshan2013semantic},
τα \newtermprint[Dialogue System]{Σύστημα Διαλόγου}~\cite{tur2005semi,chen2013unsupervised},
και την \newtermprint[Event Extraction]{Εξαγωγή Γεγονότων}~\cite{surdeanu2003using,exner2011using,rospocher2016building}.

\subsection{Συναναφορές}\label{subsec:coref}
Σε ένα κείμενο η \newterm{Συναναφορά}{Coreference} συμβαίνει όταν δύο ή περισσότερες φράσεις αναφέρονται στην ίδια οντότητα (πρόσωπο, τοποθεσία, πράγμα, κτλ).
Για παράδειγμα, στην πρόταση \enquote{Bill said he would come} το κύριο όνομα \engquote{Bill} και η αντωνυμία \engquote{he} αναφέρονται στην ίδια οντότητα, το άτομο που ονομάζεται \en{Bill}.
Αυτό είναι ένα παράδειγμα \newterm[Συναναφορά - Coreference]{Αναφορά\dd{ς}}{Anaphora} καθώς η λέξη \engquote{he} ακολουθεί το όνομα στο οποίο αναφέρεται.
Άλλος τύπος συναναφοράς αποτελεί η \newterm[Συναναφορά - Coreference]{Καταφορά}{Cataphora} όπως στην πρόταση \engquote{If they are angry about the music, the neighbors will call the cops} όπου η καταφορά \engquote{they} προηγείται της φράσης \engquote{the neighbors} στην οποία αναφέρεται.
Άλλες μορφές είναι η \newterm[Συναναφορά - Coreference]{Διάσπαση Προηγουμένων Αναφορών}{Split Antecedents}, π.χ.\ \engquote{When Carol helps Bob and Bob helps Carol, they can accomplish any task} όπου το \engquote{they} αναφέρεται ταυτόχρονα στους \engquote{Bob} και \engquote{Carol}
και η \newterm[Συναναφορά - Coreference]{Συναναφορά Ονοματικών Φράσεων}{Coreferring Noun Phrases}, π.χ.\ \engquote{The project leader is refusing to help. The jerk thinks only of himself} όπου τα \engquote{the project leader} και {the jerk} αναφέρονται στο ίδιο πρόσωπο.

Στην κατανόηση φυσικής γλώσσας, η επίλυση σχέσεων συναναφοράς συνίσταται στην αναζήτηση του ποιες λέξεις αναφέρονται στις ίδιες οντότητες σε ένα κομμάτι κειμένου.
Μπορεί να χρησιμοποιηθεί σε εφαρμογές επεξεργασίας φυσικής γλώσσας όπου απαιτείται ένα πιο υψηλό επίπεδο κατανόησης~\cite{peng2015solving}.
Μια καθολική λύση αποτελεί δύσκολο στόχο καθώς απαιτεί καλή κατανόηση του νοήματος του κειμένου και εξωτερική γνώση.
Η δοκιμασία τεχνητής νοημοσύνης Winograd~\cite{levesque2012winograd} χρησιμοποιεί συναναφορές για να δημιουργήσει προκλήσεις που δεν επιλύονται εύκολα από υπολογιστές.
Για παράδειγμα, στην πρόταση \engquote{The women stopped taking the pills because they were <$X$>} η αναφορά \engquote{they} εξαρτάται από τη λέξη $X$.
Αν $X = \text{\en{pregnant}}$ τότε αναφέρεται στις γυναίκες (\en{the women}) που είναι έγκυες.
Ωστόσο, αν $X = \text{\en{carcinogenic}}$ τότε το \engquote{they} αναφέρεται στα χάπια (\en{pills}) που είναι καρκινογόνα.

\subsection{Σημασιολογική ανάλυση}
Ο όρος \newterm{Σημασιολογική Ανάλυση}{Semantic Parsing} αναφέρεται στη διαδικασία επεξεργασίας ενός κειμένου φυσικής γλώσσας και την ανάθεση μιας σημασιολογικής δομής σε αυτό~\cite{martin2009speech}.
Μπορεί να βρει εφαρμογές
στην \newtermprint[Question Answering]{Απάντηση Ερωτήσεων}~\cite{berant2013semantic},
στην \newtermprint[Code Generation]{Παραγωγή Κώδικα}~\cite{rabinovich2017abstract,yin2017syntactic}
και στη \newtermprint[Machine Translation]{Μηχανική Μετάφραση}~\cite{andreas2013semantic}.

Συγγενικό στόχο αποτελεί η \newterm{Θεμελίωση Φυσικής Γλώσσας}{Grounding Natural Language} κατά την οποία η φυσική γλώσσα αντιστοιχίζεται σε σημασιολογικές δομές που επιτρέπουν την κατανόηση και εκτέλεση από ένα ρομπότ~\cite{matuszek2013learning}.
Επίσης, ένας άλλος σχετικός όρος είναι η \newterm{Σημασιολογική Επισημείωση}{Semantic Tagging} που αναφέρεται στη διαδικασία τοποθέτησης επιπρόσθετων πληροφοριών στα στοιχεία ενός κειμένου

\subsection{Διανύσματα λέξεων}
Τα \newterm{Διανύσματα Λέξεων}{Word Vectors} χρησιμοποιούνται για την αντιμετώπιση του \newterm{\rr{Προβλήματος}{Πρόβλημα} της Διαστασιμότητας}{Curse of Dimensionality} κατά το οποίο τα δεδομένα που βρίσκονται στον χώρο του προβλήματος γίνονται αραιά λόγω της ραγδαίας αύξησης του όγκου του χώρου.
Τα διανύσματα αυτά αντιστοιχίζουν λέξεις ή φράσεις του λεξιλογίου σε διανύσματα πραγματικών αριθμών μικρότερης διάστασης.

Σύνολα από έτοιμα προ-εκπαιδευμένα διανύσματα λέξεων προσφέρονται, μεταξύ άλλων, από τα \libcite{word2vec}, \libcite{GloVe}, \libcite{ELMO} και \libcite{BERT}.

% vim:ts=4:sw=4:expandtab:fo-=tc:tw=120
