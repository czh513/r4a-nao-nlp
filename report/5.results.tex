\chapter{Πειράματα \& Αποτελέσματα}\label{chap:results}% TODO: write blabla to fit graphs
\section{Πιθανά προβλήματα}\label{sec:problems}
\subsection{Λόγω \lib{spaCy}}\label{subsec:problems-spacy}
Καθώς τα μοντέλα της αγγλικής γλώσσας που παρέχονται από τη βιβλιοθήκη \lib{spaCy} χρησιμοποιούνται και από την υλοποίηση SRL της βιβλιοθήκης \lib{AllenNLP},
σφάλματα στην πρώτη μπορούν να οδηγήσουν σε λάθος αποτελέσματα στη δεύτερη.
Για παράδειγμα, στις προτάσεις της καταχώρησης \ref{lst:problem-POS} υπάρχουν λάθη στην \newterm{Επισημείωση Μερών του Λόγου}{Part-of-Speech Tagging}
και η εσφαλμένη ανάθεση σημασιολογικών ρόλων φαίνεται στην \fref{lst:problem-POS-SRL}.

Μια λύση για αυτό το πρόβλημα αποτελεί η επανεκπαίδευση των μοντέλων της \lib{spaCy} πάνω σε νέα δεδομένα\footnote{\url{https://explosion.ai/blog/pseudo-rehearsal-catastrophic-forgetting}}.

\begin{listing}[p] % Using p here since they fit perfectly in a page
    \begin{minted}{python}
>>> import spacy
>>> nlp = spacy.load("en_core_web_sm")  # Μοντέλο που χρησιμοποιεί ως προεπιλογή η AllenNLP
>>> # Σημειώνεται ότι η πρόσφατη νεώτερη έκδοση δεν εμφανίζει όλα τα προβλήματα που αναφέρονται εδώ
>>> nlp._meta["version"]
'2.0.0'
>>> doc = nlp("Replay the motion named XXX")
>>> doc[0].pos_  # Σωστό
'VERB'
>>> doc = nlp("Replay motion named XXX")  # Χωρίς το "the"
>>> doc[0].pos_  # Θα έπρεπε και πάλι να είναι "VERB"
'NOUN'
>>> # Προβλήματα με κατευθύνσεις "left" και "right"
>>> doc = nlp("Turn right")
>>> doc[-1].pos_  # Σωστό
'ADV'
>>> doc = nlp("Turn right and then left")
>>> doc[-1].pos_  # Θα έπρεπε και πάλι να είναι "ADV"
'VERB'
    \end{minted}
    \caption{Σφάλματα στην αντιστοίχιση ετικετών μερών του λόγου}\label{lst:problem-POS}
\end{listing}
\begin{listing}[p]
    \begin{minted}{python}
>>> from allennlp.predictors.predictor import Predictor
>>> predictor = Predictor.from_path(
... "https://s3-us-west-2.amazonaws.com/allennlp/"
... "models/srl-model-2018.05.25.tar.gz"
... )
>>> predictor = Predictor.from_path(
...     "https://s3-us-west-2.amazonaws.com/"
...     "allennlp/models/srl-model-2018.05.25.tar.gz"
... )
>>> [verb["description"] for verb in predictor.predict("Turn right")["verbs"]]
['[V: Turn] [ARGM-DIR: right]']
>>> [verb["description"] for verb in predictor.predict("Turn right and then left")["verbs"]]
['[V: Turn] [ARGM-DIR: right] and then left',
 'Turn right and [ARGM-TMP: then] [V: left]']
    \end{minted}
    \caption{Πως επηρεάζεται η υλοποίηση SRL από τα σφάλματα στην αντιστοίχιση ετικετών μερών του λόγου}\label{lst:problem-POS-SRL}
\end{listing}

\subsection{Λόγω ανάθεσης σημασιολογικών ρόλων}
Ένα παράδειγμα προβλήματος που μπορεί να προκύψει από την ανάθεση σημασιολογικών ρόλων είχε παρουσιαστεί στην \fref{lst:srl}.
Ένα δεύτερο παράδειγμα παρατίθεται στην \fref{lst:srl-problem}.

\begin{listing}[t]
    \begin{minted}{python}
>>> from allennlp.predictors.predictor import Predictor
>>> predictor = Predictor.from_path(
...     "https://s3-us-west-2.amazonaws.com/allennlp/"
...     "models/srl-model-2018.05.25.tar.gz"
... )
>>> [
...     verb["description"]
...     for verb in predictor.predict("Stand up and move forwards 5 meters")["verbs"]
... ]
['[V: Stand] up and move forwards 5 meters',
 'Stand up and [V: move] [ARGM-DIR: forwards] [ARG2: 5 meters]']
\end{minted}
    \lcaption{Σφάλμα στην ανάθεση σημασιολογικών ρόλων}{%
        Το \engquote{up} δεν συμμετέχει στη πρώτη δομή.%
    }\label{lst:srl-problem}
\end{listing}

\subsection{Λόγω επισήμανσης συναναφορών}
Όπως αναφέρθηκε, το πρόβλημα της επισήμανσης συναναφορών (\fref{subsec:coref}) θεωρείται ακόμα ανοιχτό στον χώρο της κατανόησης φυσικής γλώσσας.
Για αυτό, συναντήθηκαν διάφορες προτάσεις στις οποίες κανένα από τα μοντέλα δεν κατάφερε να βρει επιτυχώς όλες τις συναναφορές που υπήρχαν.
Σε τέτοιες προτάσεις, η μηχανή της snips μπορεί να αποτύχει να βρει όλες τις οντότητες ή ακόμα και τη σωστή πρόθεση του χρήστη.

Επίσης προβλήματα μπορούν να δημιουργηθούν όταν γίνει λάθος ομαδοποίηση των συναναφορών και μια δευτερεύουσα αναφορά αντικατασταθεί με μια πρωτεύουσα αναφορά που ανήκει σε διαφορετική οντότητα.

\subsection{Λόγω NLU}
Προβλήματα μπορούν να προκύψουν όταν το εκπαιδευμένο μοντέλο \NLU{} δεν μπορεί να βρει την πρόθεση του χρήστη στην τελική πρόταση μετά τις υπόλοιπες μετατροπές.
Συνήθως, απαιτείται η ενίσχυση της βάσης δεδομένων με περισσότερα παραδείγματα.

\section{Αποτελέσματα σε σενάρια χρήσης}
Η δοκιμή του συστήματος γίνεται με διάφορα σενάρια χρήσης.
Η έξοδος του συστήματος μορφοποιείται σε γράφο με τη βοήθεια του \libcite{Graphviz} και παρουσιάζεται στα σχήματα αυτής της ενότητας.

\subsection{Δοκιμαστικά σενάρια χρήσης}
Σε αυτή την υποενότητα παρατίθενται τα σενάρια που συγγράφηκαν κατά τη διάρκεια της υλοποίησης της διπλωματικής.
Αποτελούν παραδείγματα για τα οποία υπάρχει η προσδοκία καλής απόδοσης.
Λαμβάνουν υπόψη τους περιορισμούς του συστήματος και χρησιμοποιούνται για να επιδείξουν τις δυνατότητές του συνδυάζοντας αρκετές από τις έννοιες που έχουν αναφερθεί προηγουμένως.

\begin{figure}
    \makeatletter%
    \centering
    \def\svgscale{0.85}%
    \ig@escapeunderscore{\input{images/graphs/with-modifier.pdf_tex}}
    \lcaption{Παράδειγμα από δοκιμαστικό σενάριο χρήσης}{%
        Αρχικό κείμενο: \engquote{Recognize speech without moving}.

        Σε αυτή την πρόταση το σύστημα αναγνωρίζει την πρόθεση \intent{BodyMotion} χωρίς ορίσματα.
        Η ακινησία (\engquote{without moving}) δεν αποτελεί επιλογή του \metamodel{} σε αυτή την ενέργεια καθώς είναι η υπόρρητη προεπιλογή.
        Έτσι, ένα σύστημα που αξιοποιεί την έξοδο του \projectname{} πρέπει να αφαιρέσει τον πλεονασμό.
    }%
    \def\svgscale{0.85}
    \ig@escapeunderscore{\input{images/graphs/multi-entity-double-intent.pdf_tex}}
    \lcaption{Παράδειγμα από δοκιμαστικό σενάριο χρήσης}{%
        Αρχικό κείμενο: \engquote{Enable the leds of your chest and legs and go left}.

        Εδώ φαίνεται πως το σύστημα έχει τη δυνατότητα να ανιχνεύει πολλαπλές οντότητες σε μια πρόθεση αλλά και να διαχωρίζει δύο προθέσεις σε μια πρόταση.
    }%
    \makeatother%
\end{figure}
\begin{figure}
    \centering
    \makeatletter%
    \def\svgscale{0.85}%
    \ig@maxfigure{0.4\textheight}{%
        \ig@escapeunderscore{\input{images/graphs/replay1.pdf_tex}}}{%
        \lcaption{Παράδειγμα από δοκιμαστικό σενάριο χρήσης}{%
            Αρχικό κείμενο: \engquote{Learn motion \engquote{bakasana}. After that, replay it}.
        }}%
    \def\svgscale{0.85}%
    \ig@maxfigure{0.5\textheight}{%
        \ig@escapeunderscore{\input{images/graphs/replay2.pdf_tex}}}{%
        \lcaption{Παράδειγμα από δοκιμαστικό σενάριο χρήσης}{%
            Αρχικό κείμενο: \engquote{Learn motion \engquote{shake hips}. After that, take 5 steps forward, replay the first motion while extending your arms}.

            Εδώ το σύστημα δεν είναι ικανό να αναγνωρίσει αναφορές που βασίζονται σε λογικές διαδικασίες.
            Για την κατανόηση του \engquote{first motion} θα απαιτούνταν η ταξινόμηση των οντοτήτων σε κλάσεις και η κατανόηση της αρίθμησής τους.
            Αυτή η διαδικασία μπορεί να αναληφθεί από λογισμικό που επεξεργάζεται την έξοδο του \projectname{}.
        }}%
    \makeatother%
\end{figure}
\begin{figure}
    \centering
    \makeatletter%
    \def\svgscale{0.85}%
    \ig@maxfigure{0.4\textheight}{%
        \ig@escapeunderscore{\input{images/graphs/coref-hand.pdf_tex}}}{%
        \lcaption{Παράδειγμα από δοκιμαστικό σενάριο χρήσης}{%
            Αρχικό κείμενο: \engquote{Detect touch on your right hand while waving it}.

            Εδώ γίνεται σωστή επίλυση αναφοράς, μεταξύ διαφορετικών προθέσεων, της αντωνυμίας \engquote{it}.
        }}%
    \def\svgscale{0.85}%
    \ig@maxfigure{0.5\textheight}{%
        \ig@escapeunderscore{\input{images/graphs/right-node-raising.pdf_tex}}}{%
        \lcaption{Παράδειγμα από δοκιμαστικό σενάριο χρήσης}{%
            Αρχικό κείμενο: \engquote{Raise and open your left hand without extending it}.

            Σε αυτό το παράδειγμα το σύστημα επιτυχώς διαχωρίζει μια δύσκολη συντακτική δομή που χρησιμοποιεί την ανύψωση δεξιού κόμβου (βλέπε \fref{subsec:linguistics}).
            Η οντότητα \engquote{left arm} χρησιμοποιείται επιτυχώς και στις τρεις προθέσεις κίνησης του χεριού:
            στις δύο πρώτες ως όρισμα του ρήματος στη δομή ανύψωσης δεξιού κόμβου και στην τελευταία μέσω αναφοράς με την αντωνυμία \engquote{it}.
        }}%
    \makeatother%
\end{figure}
\ig[type=inkscape]{long-corefs}{%
    \lcaption{Παράδειγμα από δοκιμαστικό σενάριο χρήσης}{%
        Αρχικό κείμενο: \engquote{Open your left hand and then extend it while saying hello.
            Then turn left and offer it.
            If you detect a human face, turn right.
            Else, move forwards}.

        Σε αυτό το παράδειγμα παρουσιάζεται η δυνατότητα του συστήματος να βρίσκει συναναφορές που βρίσκονται σε μεγάλη απόσταση σε σχέση με τη κύρια αναφορά.
    }%
}
\ig[type=inkscape]{other-coref}{%
    \lcaption{Παράδειγμα από δοκιμαστικό σενάριο χρήσης}{%
        Αρχικό κείμενο: \engquote{Move forwards, open your left hand and turn left.
            If you see a human, close your hand.
            Else, sit down.}.

        Εδώ φαίνεται πως η αναφορά \engquote{your hand} επιλύεται και χρησιμοποιείται σωστά ως όρισμα στη δεύτερη κίνηση του χεριού.
    }%
}

\FloatBarrier
\subsection{Πραγματικά σενάρια χρήσης}
Στη συνέχεια, ελέγχθηκε η αποτελεσματικότητα του συστήματος στην παραγωγή κώδικα από σενάρια που συλλέχθηκαν από
χρήστες\footnote{Τα σενάρια που υποβλήθηκαν βρίσκονται στον σύνδεσμο \url{https://goo.gl/V5FyDH}} που δε συμμετείχαν στη συγγραφή αυτής της εργασίας.

Λόγω της δυσκολίας συλλογής μεγάλου πλήθους δεδομένων εκπαίδευσης (που αποτελεί χρονοβόρα διαδικασία η οποία θεωρήθηκε εκτός των στόχων αυτής της εργασίας),
το \projectname{} παρουσιάζει μειωμένη απόδοση σε εισόδους που είναι γραμμένες με αρκετά διαφορετικό τρόπο.
Για παράδειγμα, εμφανίζονται προβλήματα στην αναγνώριση της πρόθεσης προτάσεων που είναι γραμμένες στο τρίτο πρόσωπο (\engquote{NAO does X}) αντί για προστακτική (\engquote{Do X}).

Επίσης, όπου κρίνεται απαραίτητο, γίνονται γραμματικές, ορθογραφικές και συντακτικές διορθώσεις και τοποθετούνται εισαγωγικά.
Σε αυτή την υποενότητα παρουσιάζονται ενδεικτικά μερικά αποτελέσματα του συστήματος σε αυτά τα κείμενα εισόδου.

\ig[type=inkscape]{22}{%
    \lcaption{Παράδειγμα από πραγματικό σενάριο χρήσης}{%
        Αρχικό κείμενο: \engquote{If you see me leaving the house warn me about potential bad weather}.

        Η πρόταση αυτή είναι αρκετά γενική και περιλαμβάνει οντότητες που το σύστημα δεν αναγνωρίζει.
        Για τη σωστή ανάλυσή της θα απαιτούνταν κάποιο σύστημα υψηλότερης λογικής που θα μπορούσε να κατασκευάσει έναν αλγόριθμο από γενικές έννοιες.
    }%
}
\ig[type=inkscape]{0}{%
    \lcaption{Παράδειγμα από πραγματικό σενάριο χρήσης}{%
        Αρχικό κείμενο: \engquote{Initially, say \engquote{Hello everyone}.
            Then say the current date and time.
            At the same time, if someone touches you, turn the leds on.
            Finally, say \engquote{Nice to meet you my friend}.
        }
    }%
}
\ig[svgwidth=\linewidth,type=inkscape]{1}{%
    \lcaption{Παράδειγμα από πραγματικό σενάριο χρήσης}{%
        Αρχικό κείμενο: \engquote{Do the following actions forever:
            Initially, look right and then look left.
            At the same time, if someone says \engquote{NAO}, walk one step forward}.
    }%
}
\ig[type=inkscape]{2}{%
    \lcaption{Παράδειγμα από πραγματικό σενάριο χρήσης}{%
        Αρχικό κείμενο: \engquote{First of all, say \engquote{What do you want to ask me?}.
            Then, wait until you hear someone talking.
            If you hear someone talking, say \engquote{Please speak louder, thank you}}.
    }%
}
\ig[svgscale=0.8,type=inkscape]{3}{%
    \lcaption{Παράδειγμα από πραγματικό σενάριο χρήσης}{%
        Αρχικό κείμενο: \engquote{Look for humans, if you detect someone, whisper \engquote{I'm coming}.
            Then turn on all the leds and wait 5 seconds.
            After that, shout \engquote{Your time is due}.
            When this happens, turn the leds off}.
    }%

    Εδώ στις εκφράσεις \engquote{whisper} και \engquote{shout} αποδίδεται ιδιαίτερη σημασία μέσω των βοηθητικών προθέσεων \intent{Shout} και \intent{Whisper} που μεταφράζονται στην υπαρκτή πρόθεση \intent{Talk}.
}
\ig[svgscale=0.8,type=inkscape]{4}{%
    \lcaption{Παράδειγμα από πραγματικό σενάριο χρήσης}{%
        Αρχικό κείμενο: \engquote{Look for humans, if you detect someone, make 3 steps towards that way.
            When this happens, turn on the leds and wait 4 seconds.
            At the same time shout \engquote{Hello human}.
            When this happens, turn the leds off.
            Finally, whisper \engquote{You're doomed}}.
    }%
}
\ig[svgscale=0.8,type=inkscape]{19}{%
    \lcaption{Παράδειγμα από πραγματικό σενάριο χρήσης}{%
        Αρχικό κείμενο: \engquote{Start trying to find a human.
            Once you find one, do the following: Say \engquote{Hello mortal, let's play a luck-based game}.
            Then, say \engquote{Teach me one motion for my left hand}.
            Then, learn the 1st motion.
            Then, say \engquote{Teach me a motion for my right hand}.
            Then, learn the 2nd motion.
            Then, say \engquote{Teach me a motion for my head}'.
            Then, learn the 3rd motion.
            Once you have learned all the motions, say \engquote{Ok now guess which motion I will make}' and start listening for either \engquote{1}, \engquote{2} or \engquote{3}.
            Once you hear \engquote{1}, \engquote{2} or \engquote{3},
            make randomly one of the 3 motions you learned and if that motion is the same as the one the user said say \engquote{You won!},
            else say \engquote{You lost, I told you, you are mortal after all!}}

        Στο τέλος, η σωστή ανίχνευση των προθέσεων αποτυχαίνει καθώς απαιτείται η χρήση λογικής.
    }%
}

% vim:ts=4:sw=4:expandtab:fo-=tc:tw=120
